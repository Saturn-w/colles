\subsection{ 
Démonstration de l'espérance de la loi géométrique\\
$X \hookrightarrow \mathcal{G}(p)$. \\ }

\color{black}
\vspace{0.5cm}
Soit $X \hookrightarrow \mathcal{G}(p)$. \\ 
Montrer que $X$ est d'espérance finie et calculer $E(X)$.


\vspace{0.3cm}

\noindent $X(\Omega) = \mathbb{N}^* \quad \forall n \in \mathbb{N}^*$ \\
\noindent $P(X=n) = q^{n-1}p \quad$ où $q = 1-p$

\vspace{0.3cm}

\noindent $X$ d'esp finie $\iff \sum_n n P(X=n) \text{ CV (SATP)}$ \\
\noindent \phantom{$X$ d'esp finie} $\iff \sum_n n q^{n-1}p \text{ CV}$

\vspace{0.3cm}

\noindent On sait que \\
\noindent $\forall x \in ]-1, 1[, \quad \frac{1}{1-x} = \sum_{n=0}^{+\infty} x^n$

\vspace{0.3cm}

\noindent Par dérivation terme à terme d'une SE \\
\noindent $\frac{1}{(1-x)^2} = \sum_{n=1}^{+\infty} n x^{n-1}$

\vspace{0.3cm}

\noindent Or $q \in ]0, 1[$ \\
\noindent Donc $\sum n q^{n-1} \text{ CV}$ \quad Donc $X$ est d'esp finie

\vspace{0.3cm}

\noindent et $E(X) = \sum_{n=1}^{+\infty} n q^{n-1} p = \frac{1}{(1-q)^2} p = \frac{p}{p^2} = \frac{1}{p}$

\vspace{0.5cm}

\noindent Donc \fbox{$E(X) = \frac{1}{p}$}