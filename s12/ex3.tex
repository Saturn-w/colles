\subsection{Démonstration que $x \mapsto \text{Arctan}(x)$ est DSE}

\color{black}
\vspace{0.5cm}

\noindent $f = \text{Arctan} \in \mathcal{C}^1(\mathbb{R})$

\vspace{0.3cm}

\noindent $\forall x \in \mathbb{R}, \quad f'(x) = \frac{1}{1+x^2}$

\vspace{0.3cm}

\noindent Or $\forall t \in ]-1, 1[, \quad \frac{1}{1-t} = \sum_{n=0}^{+\infty} t^n$

\vspace{0.3cm}

\noindent Donc $\forall x \in ]-1, 1[, \quad t = -x^2$
\[ \frac{1}{1+x^2} = \sum_{n=0}^{+\infty} (-x^2)^n = \sum_{n=0}^{+\infty} (-1)^n x^{2n} \]

\vspace{0.3cm}

\noindent Donc $\forall x \in ]-1, 1[$
\[ [\text{Arctan}(t)]_0^x = \sum_{n=0}^{+\infty} (-1)^n \frac{x^{2n+1}}{2n+1} \quad (\text{par intégration terme à terme d'une série entière sur } [0, x]) \]

\vspace{0.5cm}

\noindent Donc \fbox{$\text{Arctan}(x) = \sum_{n=0}^{+\infty} (-1)^n \frac{x^{2n+1}}{2n+1}$}