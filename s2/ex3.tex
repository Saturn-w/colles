\subsection{Démonstration du Critère Spécial des Séries Alternées}

\color{black}
\vspace{0.3cm}

\noindent $\sum u_n$ est une série alternée.\\
$(|u_n|)_{n \in \mathbb{N}}$ $\searrow$ et converge vers 0.

\vspace{0.3cm}
\noindent Posons $\forall n \in \mathbb{N}, \quad a_n = S_{2n} = \sum_{k=0}^{2n} u_k$\\
\hspace*{3cm} $b_n = S_{2n+1} = \sum_{k=0}^{2n+1} u_k$

\vspace{0.3cm}
\noindent On suppose ici que $\forall n \in \mathbb{N}, \quad u_n (-1)^n \ge 0$.\\
Donc $|u_n| = (-1)^n u_n$.

\vspace{0.5cm}
\noindent \textbf{Montrons que $(a_n)$ et $(b_n)$ sont des suites adjacentes :}

\noindent Soit $n \in \mathbb{N}$,
\begin{itemize}
    \item $a_{n+1} - a_n = S_{2n+2} - S_{2n} = \sum_{k=2n+1}^{2n+2} u_k = u_{2n+2} + u_{2n+1}$
    \[ = |u_{2n+2}| - |u_{2n+1}| \]
    $\le 0$ \quad car $|u_n|$ est une suite décroissante.
    \noindent Donc $(a_n)$ est décroissante.

    \item $b_{n+1} - b_n = S_{2n+3} - S_{2n+1} = u_{2n+3} + u_{2n+2} \ge 0$
    \[ = -|u_{2n+3}| + |u_{2n+2}| \ge 0 \quad \text{car } |u_n| \searrow \]
    \noindent Donc $(b_n)$ est une suite croissante.

    \item $a_n - b_n = S_{2n} - S_{2n+1} = -u_{2n+1} = |u_{2n+1}| \xrightarrow{n \to +\infty} 0$
\end{itemize}

\vspace{0.3cm}
\noindent Donc $(a_n)$ et $(b_n)$ sont adjacentes.\\
Donc elles convergent vers la même limite $l$.\\
Donc $(S_n)$ converge.

\vspace{0.3cm}
\noindent \fbox{
    \begin{minipage}{\textwidth}
        \centering
        Donc \quad $\sum u_n$ converge.
    \end{minipage}
}