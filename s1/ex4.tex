

% Partie Titre / Question en bleu
\noindent \color{questionblue}
\subsection{ Montrer que la famille des $(n^k)_n$ (avec $k \in \mathbb{N}$) est une famille libre dans $\mathbb{R}^{\mathbb{N}}$.}

% Partie Réponse en noir
\color{black}
\vspace{0.4cm}

\noindent Posons pour tout $k \in \mathbb{N}$, $u_k = (n^k)_{n \in \mathbb{N}}$ (c'est une suite).\\
Montrons que $(u_k)_{k \in \mathbb{N}}$ est une famille libre.

\vspace{0.3cm}

\noindent Soit $N \in \mathbb{N}^*$. Montrons que $(u_0, \dots, u_N)$ est libre.

\noindent Soit $(\lambda_0, \dots, \lambda_N) \in \mathbb{R}^{N+1}$ tel que :
\[ \sum_{k=0}^N \lambda_k u_k = 0 \quad \text{(suite nulle)} \]
\textit{(C'est-à-dire que pour tout $n$, la somme vaut 0).}

\vspace{0.5cm}

\noindent \textbf{(M1) Méthode polynômiale}

\noindent On pose un polynôme $P = \sum_{k=0}^N \lambda_k X^k$.

\noindent On a $\forall n \in \mathbb{N}, \quad P(n) = 0$.
\noindent $P$ a une infinité de racines, donc $P$ est le polynôme nul.
\[ P = 0_{\mathbb{K}[X]} \]
\noindent Donc $\forall k, \quad \lambda_k = 0$.

\vspace{0.5cm}

\noindent \textbf{(M2) Méthode par l'absurde (asymptotique)}

\noindent Si les $\lambda_k$ ne sont pas tous nuls, alors on pose :
\[ k_0 = \max \{ k \in \llbracket 0, N \rrbracket \mid \lambda_k \neq 0 \} \]

\noindent On a pour tout $n \in \mathbb{N}^*$ :
\[ \lambda_0 + \lambda_1 n + \dots + \lambda_{k_0} n^{k_0} = 0 \]

\noindent On divise par le terme prépondérant $n^{k_0}$ :
\[ \frac{\lambda_0}{n^{k_0}} + \frac{\lambda_1}{n^{k_0-1}} + \dots + \lambda_{k_0} = 0 \]

\noindent Par passage à la limite quand $n \to +\infty$, tous les termes tendent vers 0 sauf le dernier :
\[ \lambda_{k_0} = 0 \]

\noindent \textbf{Absurde} par construction (car on a supposé $\lambda_{k_0} \neq 0$).

\vspace{0.5cm}

\noindent Donc la famille des $(u_k)_{k \in \mathbb{N}}$ est une famille libre car :\\
\noindent \fbox{
    \begin{minipage}{\textwidth}
        \centering
        Toutes ses familles finies sont libres.
    \end{minipage}
}

