\subsection{ Démonstration du théorème des Accroissements finis. \\
$f$ continue sur  $[a, b]$,  dérivable sur $]a, b[$ \\
Montrons que  $$\exists c \in ]a, b[, f(b) - f(a) = f'(c)(b-a)$$}

\color{black}
\vspace{0.5cm}

\noindent Posons $\varphi : x \longmapsto f(x) - f(a) - k(x-a)$

\vspace{0.2cm}
\noindent On choisit $k$ tel que $\varphi(b) = 0$ :
\noindent $\varphi(b) = 0 \implies f(b) - f(a) = k(b-a)$
\[ \implies \quad k = \frac{f(b)-f(a)}{b-a} \]

\vspace{0.5cm}

\noindent Par hypothèse sur $f$, $\varphi$ est continue sur $[a, b]$, dérivable sur $]a, b[$

\vspace{0.3cm}

\noindent $\varphi(b) = 0$ et $\varphi(a) = 0$
\noindent $\varphi(a) = \varphi(b)$ \quad D'après le théorème de Rolle
\noindent $\exists c \in ]a, b[, \quad \varphi'(c) = 0$

\vspace{0.5cm}

\noindent Or $\forall x \in ]a, b[, \quad \varphi'(x) = f'(x) - k$
\noindent Donc \quad $k = f'(c)$

\vspace{0.5cm}

\noindent Donc \quad \fbox{$\exists c \in ]a, b[, \quad f(b) - f(a) = f'(c)(b-a)$}