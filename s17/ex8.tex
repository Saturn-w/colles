\subsection{Compléter la matrice $A = \frac{1}{*} \begin{pmatrix} 6 & 3 & * \\ -2 & 6 & * \\ 3 & * & * \end{pmatrix}$ \\ en une matrice orthogonale directe à trace positive, et décrire la transformation \\ géométrique associée à $A$.}

\color{black}
\vspace{0.5cm}

\textbf{Analyse :} si $A \in SO(3)$ avec $tr(A) > 0$.

\vspace{0.3cm}

Alors $(C_1, C_2, C_3)$ forment une bond de $\mathbb{R}^3$ \\
Donc $(C_1 | C_2) = 18 - 12 + 3a = 0$ \\
$\Rightarrow a = -2$

\vspace{0.3cm}

$C_1 \wedge C_2 = C_3$ \\
Donc $\frac{1}{\alpha^2} \begin{pmatrix} 6 \\ -2 \\ 3 \end{pmatrix} \wedge \begin{pmatrix} 3 \\ 6 \\ -2 \end{pmatrix} = \frac{1}{\alpha^2} \begin{pmatrix} -14 \\ 21 \\ 42 \end{pmatrix}$

\vspace{0.3cm}

Or $\|C_1\|^2 = 1$ \\
Donc $\frac{1}{\alpha^2} (36 + 4 + 9) = \frac{1}{\alpha^2} 49 = 1 \Rightarrow 49 = \alpha^2 \Rightarrow \alpha = \pm 7$

\vspace{0.3cm}

Donc $C_3 = \frac{1}{49} \begin{pmatrix} -14 \\ 21 \\ 42 \end{pmatrix} = \frac{1}{7} \begin{pmatrix} -2 \\ 3 \\ 6 \end{pmatrix} = -\frac{1}{7} \begin{pmatrix} 2 \\ -3 \\ -6 \end{pmatrix}$

\vspace{0.3cm}

Si $\alpha = -7$ \\
$C_3 = -\frac{1}{7} \begin{pmatrix} 2 \\ -3 \\ -6 \end{pmatrix} \implies tr(A) = -\frac{6}{7}$ FAUX

\vspace{0.3cm}

Donc $\alpha = 7$ \\
\[ A = \frac{1}{7} \begin{pmatrix} 6 & 3 & -2 \\ -2 & 6 & 3 \\ 3 & -2 & 6 \end{pmatrix} \]

\vspace{0.3cm}

A matrice de rotation :

\vspace{0.3cm}

\textbf{Axe :} $AX = X$ \\
$\iff \begin{pmatrix} -1 & 3 & -2 \\ -2 & -1 & 3 \\ 3 & -2 & -1 \end{pmatrix} \begin{pmatrix} x \\ y \\ z \end{pmatrix} = \begin{pmatrix} 0 \\ 0 \\ 0 \end{pmatrix}$ \\
$\iff \begin{pmatrix} -1 & 3 & -2 \\ 0 & -7 & 7 \\ 0 & 7 & -7 \end{pmatrix} \quad \begin{matrix} L_2 \leftarrow L_2 - 2L_1 \\ L_3 \leftarrow L_3 + 3L_1 \end{matrix}$ \\
$\iff \begin{cases} x = y \\ y = z \end{cases}$

\vspace{0.3cm}

$E_1(A) = \mathbb{R} \begin{pmatrix} 1 \\ 1 \\ 1 \end{pmatrix}$. On pose $u = \begin{pmatrix} 1 \\ 1 \\ 1 \end{pmatrix}$ qui oriente l'axe.

\vspace{0.3cm}

\textbf{Angle :} \\
$tr(A) = \frac{18}{7} = 1 + 2 \cos \theta$ \\
$\Rightarrow \cos \theta = \frac{11}{14}$ \\
$\theta = \pm \text{Arccos}\left(\frac{11}{14}\right) [2\pi]$

\vspace{0.3cm}

$\text{sg} \sin \theta = \text{sg} [u, 7e_1, f(7e_1)]$ \\
$= \text{sg} \begin{vmatrix} 1 & 7 & 6 \\ 1 & 0 & -2 \\ 1 & 0 & 3 \end{vmatrix} = \text{sg} ( 7 \times \begin{vmatrix} 1 & -2 \\ 1 & 3 \end{vmatrix} ) = \text{sg} (-6 \times 7) < 0$

\vspace{0.5cm}

Donc \fbox{$A$ est la rotation d'angle $-\text{Arccos}\left(\frac{11}{14}\right)$ et d'axe $\begin{pmatrix} 1 \\ 1 \\ 1 \end{pmatrix}$}