\subsection{ Démonstration du déterminant de Vandermonde }

\color{black}
\vspace{0.5cm}
$(a_1, \dots, a_n) \in \mathbb{K}^n$
\[ V(a_1, \dots, a_n) = \begin{vmatrix} 1 & \dots & 1 \\ a_1 & \dots & a_n \\ \vdots & & \vdots \\ a_1^{n-1} & \dots & a_n^{n-1} \end{vmatrix} \]

\vspace{0.5cm}

\noindent \underline{Cas :} Les $a_1, \dots, a_n$ ne sont pas 2 à 2 différents

\noindent alors il y a au moins 2 colonnes identiques

\noindent Donc $V(a_1, \dots, a_n) = 0$

\vspace{0.2cm}

\noindent De plus $\exists i \neq j$ tel que $a_i = a_j$
\noindent Donc $\prod_{1 \le i < j \le n} (a_j - a_i) = 0$

\vspace{0.5cm}

\noindent \underline{Cas :} Les $a_1, \dots, a_n$ sont 2 à 2 différents

\vspace{0.3cm}

\noindent Montrons par récurrence la propriété
\[ \mathcal{P}(n) : "V(a_1, \dots, a_n) = \prod_{1 \le i < j \le n} (a_j - a_i)" \]

\vspace{0.3cm}

\noindent \underline{Initialisation :} Pour $n=2$
\[ V(a_1, a_2) = \begin{vmatrix} 1 & 1 \\ a_1 & a_2 \end{vmatrix} = a_2 - a_1 \]
\noindent $\mathcal{P}(2)$ vérifié.

\vspace{0.5cm}

\noindent \underline{Hérédité :} On suppose la propriété vérifiée pour un certain $n \ge 2$

\vspace{0.2cm}

\noindent Posons $P(x) = V(a_1, \dots, a_n, x) = \begin{vmatrix} 1 & \dots & 1 & 1 \\ a_1 & \dots & a_n & x \\ \vdots & & \vdots & \vdots \\ a_1^{n} & \dots & a_n^{n} & x^{n} \end{vmatrix}$

\vspace{0.3cm}

\noindent On développe par rapport à la dernière colonne
\[ P(x) = \sum_{i=1}^{n+1} (-1)^{(n+1)+i} x^{i-1} \det M_{i, n+1} \]
\noindent (la matrice sans la $i^e$ ligne et $n+1^e$ colonne)

\vspace{0.3cm}

\noindent $\to P$ est un polynôme en $x$ de degré $n$
\noindent avec pour coefficient dominant $1 \times \det M_{n+1, n+1} = V(a_1, \dots, a_n)$

\vspace{0.5cm}

\noindent De plus on sait que $\forall j \in \llbracket 1, n \rrbracket, \quad P(a_j) = V(a_1, \dots, a_n, a_j) = 0$
\noindent (car 2 colonnes identiques)

\vspace{0.3cm}

\noindent Donc $P$ a $n$ racines différentes, or $P$ est de degré $n$.
\noindent Donc $P$ est scindé.

\vspace{0.2cm}

\noindent D'après l'hypothèse de récurrence $V(a_1, \dots, a_n) = \prod_{1 \le i < j \le n} (a_j - a_i)$

\vspace{0.3cm}

\noindent Donc $P(x) = cd(P) \times \prod_{j=1}^n (x - a_j)$

\vspace{0.3cm}

\noindent Donc $P(a_{n+1}) = V(a_1, \dots, a_{n+1}) = \prod_{1 \le i < j \le n+1} (a_j - a_i)$