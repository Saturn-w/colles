\subsection{Une urne contient au départ une boule blanche. On joue indéfiniment à pile ou face \\ 
avec une pièce équilibrée. Chaque fois que l'on obtient face, on rajoute une boule \\ 
noire dans l'urne. Et la première fois que l'on obtient pile, on tire une boule dans l'urne. \\ 
On pose $\forall n \in \mathbb{N}^*$, $P_n$ "on obtient pile pour la première fois au $n^e$ lancer" \\
a) Déterminer un système quasi-complet d'événements. \\
b) Quelle est la probabilité de sortir une boule blanche ?}

\color{black}
\vspace{0.5cm}

\noindent \textbf{a) Montrons que $(P_n)_{n \in \mathbb{N}^*}$ est un système quasi-complet}

\vspace{0.3cm}

\noindent $\rightarrow$ Les $P_n$ sont 2 à 2 incompatibles : \\
\noindent $(\forall n \neq n') \enspace P_n \cap P_{n'} = \emptyset$ car il existe une seule "première fois".

\vspace{0.3cm}

\noindent $\rightarrow$ Montrons que $\sum_{n \in \mathbb{N}^*} P(P_n) = 1$ :
\\
\noindent On pose $\forall k \in \mathbb{N}^*$, $f_k$ : "on obtient face au $k$-ième lancer".
\noindent $\forall n \in \mathbb{N}^*$, \quad $P_n = \left( \bigcap_{k=1}^{n-1} f_k \right) \cap \overline{f_n}$

\vspace{0.3cm}

\noindent Donc \quad $P(P_n) = \prod_{k=1}^{n-1} \left( \frac{1}{2} \right) \times \left( \frac{1}{2} \right) = \frac{1}{2^n}$

\vspace{0.3cm}

\noindent D'où \quad $\sum_{n=1}^{+\infty} P(P_n) = \sum_{n=1}^{+\infty} \frac{1}{2^n}$ \quad (Série géométrique)
\[ = \frac{1}{2} \times \frac{1}{1 - \frac{1}{2}} = 1 \]

\vspace{0.3cm}

\noindent Donc \fbox{$(P_n)_n$ est un système quasi-complet d'événements.}

\vspace{0.8cm}

\noindent \textbf{b) On pose $A$ : "on tire une boule blanche". Calculons $P(A)$ :}

\vspace{0.3cm}

\noindent On utilise la formule des probabilités totales :
\[ P(A) = \sum_{n=1}^{+\infty} P(P_n) P_{P_n}(A) \]

\vspace{0.3cm}

\noindent $\rightarrow$ Si $P_n$ est réalisé, alors l'urne contient $\begin{cases} n-1 & \text{boules noires} \\ 1 & \text{boule blanche} \end{cases}$
\noindent Donc \quad $P_{P_n}(A) = \frac{1}{n}$

\vspace{0.3cm}

\noindent Donc \quad $P(A) = \sum_{n=1}^{+\infty} \frac{1}{n} \left( \frac{1}{2} \right)^n$

\vspace{0.3cm}

\noindent Or on sait que $\forall x \in ]-1, 1[$, \quad $\ln(1-x) = -\sum_{n=1}^{+\infty} \frac{x^n}{n}$

\vspace{0.3cm}

\noindent Donc \quad $P(A) = -\ln\left(1 - \frac{1}{2}\right) = \ln(2)$

\vspace{0.5cm}

\noindent D'où :
\[ \fbox{$P(A) = \ln(2)$} \]