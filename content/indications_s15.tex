\documentclass[a4paper,11pt]{article}
\usepackage[utf8]{inputenc}
\usepackage[T1]{fontenc}
\usepackage{amsmath, amssymb, stmaryrd}
\usepackage{xcolor}
\usepackage[margin=2.5cm]{geometry}
\usepackage{fancybox}
\usepackage{needspace}

\usepackage{tocloft}
\setlength{\cftbeforesecskip}{0.9cm}
\setlength{\cftbeforesubsecskip}{0.05cm}

\usepackage[nobottomtitles*]{titlesec}
\renewcommand{\bottomtitlespace}{2cm}

\definecolor{questionblue}{RGB}{0, 0, 205}

\pdfstringdefDisableCommands{%
  \def\int{∫}%
  \def\sum{∑}%
  \def\prod{∏}%
  \def\infty{∞}%
  \def\partial{∂}%
  \def\leq{≤}%
  \def\geq{≥}%
  \def\neq{≠}%
  \def\approx{≈}%
  \def\times{×}%
  \def\cdot{·}%
  \def\mathbb#1{#1}%
  \def\mathcal#1{#1}%
  \def\vec#1{#1}%
}

\newcommand{\sectionbreak}{\needspace{6\baselineskip}}
\newcommand{\subsectionbreak}{\needspace{10\baselineskip}}
\newcommand{\subsubsectionbreak}{\needspace{5\baselineskip}}

\titleformat{\section}
  {\normalfont\Large\bfseries\color{black}}{\thesection}{1em}{}
\titleformat{\subsection}
  {\normalfont\large\bfseries\color{questionblue}}{\thesubsection}{1em}{}
\renewcommand{\thesubsubsection}{\alph{subsubsection})}
\titleformat{\subsubsection}
  {\normalfont\large\bfseries\color{questionblue}}{\thesubsubsection}{0.5em}{}

\renewcommand{\thesubsection}{\arabic{subsection}}
\setcounter{tocdepth}{2}

\usepackage{fancyhdr}
\pagestyle{fancy}
\fancyhf{}
\fancyhead[L]{Indications - Semaine 15 - Inégalités probabilistes}
\fancyfoot[C]{\thepage}
\renewcommand{\headrulewidth}{0pt}

\usepackage{hyperref}
\hypersetup{hidelinks}

\begin{document}

\title{Indications -- Semaine 15 -- Inégalités probabilistes et fonctions génératrices\\
\large PSI}
\date{}
\maketitle
\tableofcontents

\clearpage

% -----------------------------------------------------------------------
\subsection{Variance de la loi de Poisson $\mathcal{P}(\lambda)$}

\medskip
\textbf{Indications.}
\begin{itemize}
  \item Calculer $E(X^2) = E(X(X-1)) + E(X)$. Calculer $E(X(X-1)) = \sum_{n=2}^{+\infty} n(n-1) \frac{\lambda^n e^{-\lambda}}{n!} = \lambda^2$.
  \item On obtient $E(X^2) = \lambda^2 + \lambda$, puis $V(X) = E(X^2) - (E(X))^2 = \lambda^2 + \lambda - \lambda^2 = \lambda$.
  \item Méthode alternative : partir de $x e^x = \sum_{k \geq 1} \frac{k x^k}{(k-1)!}$, dériver une seconde fois.
\end{itemize}

\clearpage

% -----------------------------------------------------------------------
\subsection{Inégalité de Cauchy-Schwarz pour les variables aléatoires}

Montrer $[E(XY)]^2 \leq E(X^2) E(Y^2)$.

\medskip
\textbf{Indications.}
\begin{itemize}
  \item Considérer le polynôme $P(\lambda) = E\!\left[(\lambda X + Y)^2\right] = E(X^2)\lambda^2 + 2E(XY)\lambda + E(Y^2) \geq 0$ pour tout $\lambda \in \mathbb{R}$.
  \item Un polynôme du second degré toujours $\geq 0$ a un discriminant $\leq 0$ :
  \[ \Delta = 4[E(XY)]^2 - 4 E(X^2) E(Y^2) \leq 0. \]
  \item Cas d'égalité : $\Delta = 0 \iff \lambda X + Y = 0$ p.s. pour un certain $\lambda$, i.e. $X$ et $Y$ sont p.s. proportionnelles.
\end{itemize}

\clearpage

% -----------------------------------------------------------------------
\subsection{Inégalité de Markov}

Soit $X \geq 0$ une variable aléatoire d'espérance finie. Montrer $P(X \geq a) \leq E(X)/a$ pour $a > 0$.

\medskip
\textbf{Indications.}
\begin{itemize}
  \item Écrire $E(X) = \sum_{x \geq 0} x P(X = x)$ (cas discret) ou $E(X) = \int x\, dP$.
  \item Séparer la somme : $E(X) = \sum_{x \geq a} x P(X=x) + \sum_{x < a} x P(X=x) \geq a \sum_{x \geq a} P(X=x) = a P(X \geq a)$.
  \item Diviser par $a > 0$.
\end{itemize}

\clearpage

% -----------------------------------------------------------------------
\subsection{$1/E(X) \leq E(1/X)$ pour $X > 0$}

\medskip
\textbf{Indications.}
\begin{itemize}
  \item[(a)] Si $X$ est bornée par $[m, M]$ avec $m > 0$ : $1/X$ est bien définie et bornée. Vérifier que $E(1/X)$ existe.
  \item[(b)] Cas $X \sim \mathcal{G}(p)$ : calculer $E(1/X) = \sum_{k=1}^{+\infty} \frac{1}{k} p(1-p)^{k-1}$, reconnaître $-\ln(1-q)/q$ avec $q = 1-p$. Vérifier l'inégalité $-\ln(1-q)/q \geq 1/(1-q) \cdot 1 = p/q \cdot$ ... utiliser $-\ln(1-q) \geq q$.
  \item[(c)] Cas général : appliquer C-S avec $Y = \sqrt{X}$, $Z = 1/\sqrt{X}$ : $[E(YZ)]^2 \leq E(Y^2)E(Z^2)$, i.e. $1 = [E(1)]^2 \leq E(X) \cdot E(1/X)$.
\end{itemize}

\clearpage

% -----------------------------------------------------------------------
\subsection{Bienaymé-Tchebychev appliqué}

$X_n \sim \mathcal{B}(n, p)$. Trouver $n$ tel que $P(|X_n/n - p| \geq 0{,}001) \leq 10^{-3}$.

\medskip
\textbf{Indications.}
\begin{itemize}
  \item $X_n/n$ a pour variance $V(X_n/n) = pq/n$ (avec $q = 1-p$).
  \item Bienaymé-Tchebychev : $P(|X_n/n - p| \geq \varepsilon) \leq \frac{V(X_n/n)}{\varepsilon^2} = \frac{pq}{n \varepsilon^2}$.
  \item On veut $\frac{pq}{n \varepsilon^2} \leq 10^{-3}$ avec $\varepsilon = 10^{-3}$. Majorer $pq \leq 1/4$, obtenir $n \geq \frac{pq}{\varepsilon^2 \times 10^{-3}} \leq \frac{10^6}{4} = 250000$.
\end{itemize}

\clearpage

% -----------------------------------------------------------------------
\subsection{Fonctions génératrices d'une loi custom}

$P(X = n) = \frac{x^{2n}}{\mathrm{ch}(x)(2n)!}$ pour $n \in \mathbb{N}$.

\medskip
\textbf{Indications.}
\begin{itemize}
  \item[(a)] Vérifier que c'est une loi : $\sum_{n \geq 0} \frac{x^{2n}}{(2n)!} = \mathrm{ch}(x)$, donc la somme vaut 1.
  \item Fonction génératrice : $G_X(t) = \sum_{n \geq 0} t^n P(X = n) = \frac{1}{\mathrm{ch}(x)} \sum_{n \geq 0} \frac{(x\sqrt{t})^{2n}}{(2n)!} = \frac{\mathrm{ch}(x\sqrt{t})}{\mathrm{ch}(x)}$ pour $t \geq 0$.
  \item[(b)] $E(X) = G_X'(1)$ : dériver $G_X$ et évaluer en $t = 1$. Pour la variance, utiliser $V(X) = G_X''(1) + G_X'(1) - [G_X'(1)]^2$.
\end{itemize}

\end{document}
