\documentclass[a4paper,11pt]{article}
\usepackage[utf8]{inputenc}
\usepackage[T1]{fontenc}
\usepackage{amsmath, amssymb, stmaryrd}
\usepackage{xcolor}
\usepackage[margin=2.5cm]{geometry}
\usepackage{fancybox}
\usepackage{needspace}

\usepackage{tocloft}
\setlength{\cftbeforesecskip}{0.9cm}
\setlength{\cftbeforesubsecskip}{0.05cm}

\usepackage[nobottomtitles*]{titlesec}
\renewcommand{\bottomtitlespace}{2cm}

\definecolor{questionblue}{RGB}{0, 0, 205}

\pdfstringdefDisableCommands{%
  \def\int{∫}%
  \def\sum{∑}%
  \def\prod{∏}%
  \def\infty{∞}%
  \def\partial{∂}%
  \def\leq{≤}%
  \def\geq{≥}%
  \def\neq{≠}%
  \def\approx{≈}%
  \def\times{×}%
  \def\cdot{·}%
  \def\mathbb#1{#1}%
  \def\mathcal#1{#1}%
  \def\vec#1{#1}%
}

\newcommand{\sectionbreak}{\needspace{6\baselineskip}}
\newcommand{\subsectionbreak}{\needspace{10\baselineskip}}
\newcommand{\subsubsectionbreak}{\needspace{5\baselineskip}}

\titleformat{\section}
  {\normalfont\Large\bfseries\color{black}}{\thesection}{1em}{}
\titleformat{\subsection}
  {\normalfont\large\bfseries\color{questionblue}}{\thesubsection}{1em}{}
\renewcommand{\thesubsubsection}{\alph{subsubsection})}
\titleformat{\subsubsection}
  {\normalfont\large\bfseries\color{questionblue}}{\thesubsubsection}{0.5em}{}

\renewcommand{\thesubsection}{\arabic{subsection}}
\setcounter{tocdepth}{2}

\usepackage{fancyhdr}
\pagestyle{fancy}
\fancyhf{}
\fancyhead[L]{Indications - Semaine 5 - Théorème des accroissements finis}
\fancyfoot[C]{\thepage}
\renewcommand{\headrulewidth}{0pt}

\usepackage{hyperref}
\hypersetup{hidelinks}

\begin{document}

\title{Indications -- Semaine 5 -- Théorème des accroissements finis\\
\large PSI}
\date{}
\maketitle

\clearpage

% -----------------------------------------------------------------------
\subsection{Preuve du théorème des accroissements finis}

Soit $f : [a, b] \to \mathbb{R}$ continue sur $[a,b]$, dérivable sur $]a, b[$. Montrer qu'il existe $c \in {]a,b[}$ tel que $f(b) - f(a) = f'(c)(b-a)$.

\medskip
\textbf{Indications.}
\begin{itemize}
  \item Introduire la fonction auxiliaire $\varphi(x) = f(x) - f(a) - k(x - a)$ où $k = \dfrac{f(b) - f(a)}{b - a}$.
  \item Vérifier que $\varphi(a) = \varphi(b) = 0$ : $\varphi$ est continue sur $[a,b]$, dérivable sur $]a,b[$, et s'annule aux deux extrémités.
  \item Appliquer le \textbf{théorème de Rolle} : il existe $c \in {]a,b[}$ tel que $\varphi'(c) = 0$.
  \item $\varphi'(c) = f'(c) - k = 0$ donne $f'(c) = k = \dfrac{f(b)-f(a)}{b-a}$.
\end{itemize}

\end{document}
