\documentclass[a4paper,11pt]{article}
\usepackage[utf8]{inputenc}
\usepackage[T1]{fontenc}
\usepackage{amsmath, amssymb, stmaryrd}
\usepackage{xcolor}
\usepackage[margin=2.5cm]{geometry}
\usepackage{fancybox}
\usepackage{needspace}

\usepackage{tocloft}
\setlength{\cftbeforesecskip}{0.9cm}
\setlength{\cftbeforesubsecskip}{0.05cm}

\usepackage[nobottomtitles*]{titlesec}
\renewcommand{\bottomtitlespace}{2cm}

\definecolor{questionblue}{RGB}{0, 0, 205}

\hypersetup{hidelinks}

\pdfstringdefDisableCommands{%
  \def\int{∫}%
  \def\sum{∑}%
  \def\prod{∏}%
  \def\infty{∞}%
  \def\partial{∂}%
  \def\leq{≤}%
  \def\geq{≥}%
  \def\neq{≠}%
  \def\approx{≈}%
  \def\times{×}%
  \def\cdot{·}%
  \def\mathbb#1{#1}%
  \def\mathcal#1{#1}%
  \def\vec#1{#1}%
}

\newcommand{\sectionbreak}{\needspace{6\baselineskip}}
\newcommand{\subsectionbreak}{\needspace{10\baselineskip}}
\newcommand{\subsubsectionbreak}{\needspace{5\baselineskip}}

\titleformat{\section}
  {\normalfont\Large\bfseries\color{black}}{\thesection}{1em}{}
\titleformat{\subsection}
  {\normalfont\large\bfseries\color{questionblue}}{\thesubsection}{1em}{}
\renewcommand{\thesubsubsection}{\alph{subsubsection})}
\titleformat{\subsubsection}
  {\normalfont\large\bfseries\color{questionblue}}{\thesubsubsection}{0.5em}{}

\renewcommand{\thesubsection}{\arabic{subsection}}
\setcounter{tocdepth}{2}

\usepackage{fancyhdr}
\pagestyle{fancy}
\fancyhf{}
\fancyhead[L]{Indications - Semaine 1 - Algèbre linéaire}
\fancyfoot[C]{\thepage}
\renewcommand{\headrulewidth}{0pt}

\usepackage{hyperref}
\hypersetup{hidelinks}

\begin{document}

\title{Indications -- Semaine 1 -- Algèbre linéaire\\
\large PSI}
\date{}
\maketitle
\tableofcontents

\clearpage

% -----------------------------------------------------------------------
\subsection{Base de $\mathbb{K}_n[X]$ via les polynômes de Lagrange}

Soient $a_0, \dots, a_n$ des scalaires distincts. On considère les polynômes de Lagrange
$L_i = \prod_{j \neq i} \frac{X - a_j}{a_i - a_j}$ et l'application $\psi : P \mapsto (P(a_0), \dots, P(a_n))$.

\medskip
\textbf{Indications.}
\begin{itemize}
  \item Montrer que $\psi : \mathbb{K}_n[X] \to \mathbb{K}^{n+1}$ est un isomorphisme (injectivité : un polynôme de degré $\leq n$ ayant $n+1$ racines est nul).
  \item En déduire que $(L_0, \dots, L_n)$ est une base : il y a $n+1$ polynômes dans $\mathbb{K}_n[X]$ qui est de dimension $n+1$, et ils sont linéairement indépendants (leur image par $\psi$ est la base canonique de $\mathbb{K}^{n+1}$).
\end{itemize}

\clearpage

% -----------------------------------------------------------------------
\subsection{Endomorphisme nilpotent $\Rightarrow f^p = 0$ avec $p \leq n$}

Soit $f$ un endomorphisme nilpotent d'un espace de dimension $n$.

\medskip
\textbf{Indications.}
\begin{itemize}
  \item Poser $p = \min\{k \in \mathbb{N} \mid f^k = 0\}$. Il existe donc $x$ tel que $f^{p-1}(x) \neq 0$.
  \item Montrer que la famille $(x, f(x), \dots, f^{p-1}(x))$ est libre : si $\sum \lambda_k f^k(x) = 0$, appliquer $f^{p-1-k_0}$ à cette relation pour isoler le coefficient $\lambda_{k_0}$.
  \item Une famille libre a au plus $n$ vecteurs, donc $p \leq n$.
\end{itemize}

\clearpage

% -----------------------------------------------------------------------
\subsection{Théorème du rang}

Soit $f : E \to F$ linéaire avec $E$ de dimension finie $n$.

\medskip
\textbf{Indications.}
\begin{itemize}
  \item Choisir un supplémentaire $G$ de $\ker f$ dans $E$ (possible en dimension finie).
  \item Montrer que $f_{|G} : G \to \mathrm{Im}\, f$ est un isomorphisme (injectivité directe car $\ker f \cap G = \{0\}$, surjectivité par définition de $\mathrm{Im}\, f$).
  \item Conclure : $\dim G = \dim(\mathrm{Im}\, f)$, et $\dim E = \dim(\ker f) + \dim G$.
\end{itemize}

\clearpage

% -----------------------------------------------------------------------
\subsection{Liberté de la famille $((n^k)_{n \geq 1})_{0 \leq k \leq p}$ dans $\mathbb{R}^\mathbb{N}$}

Montrer que les suites $(1)$, $(n)$, $(n^2)$, \dots, $(n^p)$ sont linéairement indépendantes dans $\mathbb{R}^\mathbb{N}$.

\medskip
\textbf{Indications (deux méthodes).}
\begin{itemize}
  \item \textbf{Méthode 1 :} Supposer $\sum_{k=0}^p \lambda_k n^k = 0$ pour tout $n \geq 1$. Poser $P = \sum \lambda_k X^k \in \mathbb{R}[X]$. Ce polynôme a une infinité de racines, donc $P = 0$, donc tous les $\lambda_k$ sont nuls.
  \item \textbf{Méthode 2 :} Diviser la relation par $n^p$ et passer à la limite $n \to +\infty$ pour obtenir $\lambda_p = 0$, puis recommencer par récurrence descendante.
\end{itemize}

\clearpage

% -----------------------------------------------------------------------
\subsection{$\mathrm{Im}\, f \oplus \ker g = E$}

Soient $f, g$ des endomorphismes de $E$ vérifiant $fg = gf$, $fgf = f$ et $gfg = g$.

\medskip
\textbf{Indications.}
\begin{itemize}
  \item \textbf{Méthode 1 (analyse-synthèse) :} Pour montrer que tout $x \in E$ s'écrit $x = x_1 + x_2$ avec $x_1 \in \mathrm{Im}\, f$ et $x_2 \in \ker g$, poser $x_1 = fg(x)$ et $x_2 = x - fg(x)$. Vérifier que $g(x_2) = 0$ en utilisant $gfg = g$.
  \item Pour l'intersection : si $x \in \mathrm{Im}\, f \cap \ker g$, écrire $x = f(y)$ et calculer $x = fgf(y) = fg(x) = f(0) = 0$.
  \item \textbf{Méthode 2 :} Montrer que $p = fg$ est un projecteur ($p^2 = p$), puis $\mathrm{Im}\, p = \mathrm{Im}\, f$ et $\ker p = \ker g$.
\end{itemize}

\clearpage

% -----------------------------------------------------------------------
\subsection{$\mathcal{D}_n(\mathbb{R}) = \mathrm{Vect}(D^k,\, 0 \leq k \leq n-1)$}

$\mathcal{D}_n(\mathbb{R})$ désigne l'espace des matrices diagonales $n \times n$. On note $D = \mathrm{diag}(1, 2, \dots, n)$.

\medskip
\textbf{Indications.}
\begin{itemize}
  \item Les matrices $I, D, D^2, \dots, D^{n-1}$ sont diagonales (vérifier).
  \item Montrer qu'elles sont linéairement indépendantes : la relation $\sum \lambda_k D^k = 0$ équivaut à $\sum \lambda_k j^k = 0$ pour $j = 1, \dots, n$, ce qui est un système de Vandermonde de déterminant non nul.
  \item Conclusion : $n$ vecteurs libres dans $\mathcal{D}_n(\mathbb{R})$ (de dimension $n$) forment une base.
\end{itemize}

\clearpage

% -----------------------------------------------------------------------
\subsection{CNS pour que $M \mapsto -\varphi(M)A + M$ soit un automorphisme}

$\varphi : \mathcal{M}_n(\mathbb{K}) \to \mathbb{K}$ est une forme linéaire non nulle, $A$ une matrice fixée. On pose $f(M) = M - \varphi(M)A$.

\medskip
\textbf{Indications.}
\begin{itemize}
  \item Calculer $\ker f$ : si $f(M) = 0$ alors $M = \varphi(M) A$. Si $\varphi(M) = 0$ alors $M = 0$. Si $\varphi(M) \neq 0$, alors $A \in \ker f$ implique $\varphi(A) \neq 0$, et on trouve $M = \lambda A$.
  \item Montrer que $\ker f \neq \{0\} \iff \varphi(A) = 1$.
  \item Conclusion : $f$ est un automorphisme $\iff \varphi(A) \neq 1$.
\end{itemize}

\clearpage

% -----------------------------------------------------------------------
\subsection{Formule de quadrature}

On cherche $w_0, \dots, w_n$ tels que $\int_a^b P(t)\, dt = \sum_{k=0}^n w_k P(a_k)$ pour tout $P \in \mathbb{K}_n[X]$.

\medskip
\textbf{Indications.}
\begin{itemize}
  \item \textbf{Méthode 1 :} Décomposer $P$ dans la base de Lagrange : $P = \sum_{k=0}^n P(a_k) L_k$. Intégrer : $\int_a^b P = \sum P(a_k) \int_a^b L_k$. Donc $w_k = \int_a^b L_k(t)\, dt$.
  \item \textbf{Méthode 2 :} Montrer que les formes linéaires $f_k : P \mapsto P(a_k)$ forment une base du dual de $\mathbb{K}_n[X]$, puis exprimer $P \mapsto \int_a^b P$ dans cette base.
  \item Les poids $w_k$ sont ainsi explicites et la formule est exacte sur $\mathbb{K}_n[X]$.
\end{itemize}

\clearpage

% -----------------------------------------------------------------------
\subsection{$\dim(H_1 \cap H_2) = n - 2$ pour deux hyperplans distincts}

$H_1$ et $H_2$ sont deux hyperplans distincts d'un espace $E$ de dimension $n$.

\medskip
\textbf{Indications.}
\begin{itemize}
  \item Appliquer la formule de Grassmann : $\dim(H_1 + H_2) = \dim H_1 + \dim H_2 - \dim(H_1 \cap H_2)$.
  \item Montrer que $H_1 + H_2 = E$ : puisque $H_1 \neq H_2$, il existe $x \in H_2 \setminus H_1$. Comme $H_1$ est un hyperplan et $x \notin H_1$, on a $E = H_1 \oplus \mathrm{Vect}(x) \subset H_1 + H_2$.
  \item Donc $\dim(H_1 \cap H_2) = (n-1) + (n-1) - n = n-2$.
\end{itemize}

\end{document}
