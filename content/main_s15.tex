\documentclass[a4paper,11pt]{article}
\usepackage[utf8]{inputenc}
\usepackage[T1]{fontenc}
\usepackage{amsmath, amssymb, stmaryrd}
\usepackage{xcolor}
\usepackage[margin=2.5cm]{geometry}
\usepackage{fancybox}
\usepackage{needspace}

\usepackage{tocloft}
\setlength{\cftbeforesecskip}{0.9cm} % Ajuste la valeur (ex: 1em, 15pt, etc.)

\setlength{\cftbeforesubsecskip}{0.05cm}

\usepackage[nobottomtitles*]{titlesec} % On ajoute l'option à l'import
\renewcommand{\bottomtitlespace}{2cm}

% Définition des couleurs
\definecolor{questionblue}{RGB}{0, 0, 205}

%cache les rectangles rouge sde sliens

% Sécurité : Définit l'espace nécessaire AVANT chaque titre
\newcommand{\sectionbreak}{\needspace{6\baselineskip}}
\newcommand{\subsectionbreak}{\needspace{10\baselineskip}}
\newcommand{\subsubsectionbreak}{\needspace{5\baselineskip}}


% Configuration des titres pour correspondre à ton style (1.1 en Bleu)
% Sections (Semaines)
\titleformat{\section}
  {\normalfont\Large\bfseries\color{black}}{\thesection}{1em}{}
  % Subsections (Exercices) 
\titleformat{\subsection}
  {\normalfont\large\bfseries\color{questionblue}}{\thesubsection}{1em}{}
% Subsubsections (Questions) 
\renewcommand{\thesubsubsection}{\alph{subsubsection})}
\titleformat{\subsubsection}
  {\normalfont\large\bfseries\color{questionblue}}{\thesubsubsection}{0.5em}{}


%\pretocmd{\section}{\addtocontents{toc}{\protect\filbreak}}{}{}



% Redéfinition de la numérotation des sous-sections
\renewcommand{\thesubsection}{\arabic{subsection}}
\setcounter{tocdepth}{2}

% Hyperref chargé en dernier
\usepackage{hyperref}
\hypersetup{hidelinks}

% Définir les symboles Unicode pour les bookmarks PDF
\pdfstringdefDisableCommands{%
  \def\int{∫}%
  \def\sum{∑}%
  \def\prod{∏}%
  \def\infty{∞}%
  \def\partial{∂}%
  \def\leq{≤}%
  \def\geq{≥}%
  \def\neq{≠}%
  \def\approx{≈}%
  \def\times{×}%
  \def\cdot{·}%
  \def\mathbb#1{#1}%
  \def\mathcal#1{#1}%
  \def\vec#1{#1}%
}
\begin{document}

\title{Semaine 15 -  Variables aléatoires discrètes\\
\large PSI}
\date{}
\maketitle
\tableofcontents

\clearpage \subsection{ Démonstration du théorème des Accroissements finis. \\
$f$ continue sur  $[a, b]$,  dérivable sur $]a, b[$ \\
Montrons que  $$\exists c \in ]a, b[, f(b) - f(a) = f'(c)(b-a)$$}

\color{black}
\vspace{0.5cm}

\noindent Posons $\varphi : x \longmapsto f(x) - f(a) - k(x-a)$

\vspace{0.2cm}
\noindent On choisit $k$ tel que $\varphi(b) = 0$ :
\noindent $\varphi(b) = 0 \implies f(b) - f(a) = k(b-a)$
\[ \implies \quad k = \frac{f(b)-f(a)}{b-a} \]

\vspace{0.5cm}

\noindent Par hypothèse sur $f$, $\varphi$ est continue sur $[a, b]$, dérivable sur $]a, b[$

\vspace{0.3cm}

\noindent $\varphi(b) = 0$ et $\varphi(a) = 0$
\noindent $\varphi(a) = \varphi(b)$ \quad D'après le théorème de Rolle
\noindent $\exists c \in ]a, b[, \quad \varphi'(c) = 0$

\vspace{0.5cm}

\noindent Or $\forall x \in ]a, b[, \quad \varphi'(x) = f'(x) - k$
\noindent Donc \quad $k = f'(c)$

\vspace{0.5cm}

\noindent Donc \quad \fbox{$\exists c \in ]a, b[, \quad f(b) - f(a) = f'(c)(b-a)$}
\clearpage \subsection{ Montrer que si $p$ projecteur alors $tr(p) = rg(p)$}

\color{black}
\vspace{0.5cm}

\noindent $p$ projecteur Donc $Im p \oplus \ker(p) = E$

\vspace{0.5cm}

\noindent Posons $r = rg(p) = \dim(Im(p))$
\noindent Soit $\mathcal{B} = (e_1, \dots, e_r, e_{r+1}, \dots, e_n)$ une base adaptée à cette décomposition.

\vspace{0.5cm}

\noindent Alors $mat_{\mathcal{B}}(p) = \left( \begin{array}{ccc|c} 1 & & 0 & \\ & \ddots & & 0 \\ 0 & & 1 & \\ \hline & 0 & & 0 \end{array} \right)$

\vspace{0.3cm}

\noindent $\forall j \in \llbracket 1, r \rrbracket, \quad e_j \in Im(p) \implies p(e_j) = e_j$
\noindent $\forall j \in \llbracket r+1, n \rrbracket, \quad e_j \in \ker(p) \implies p(e_j) = 0$

\vspace{0.5cm}

\noindent $mat_{\mathcal{B}}(p) = \left( \begin{array}{c|c} I_r & 0 \\ \hline 0 & 0 \end{array} \right)$

\vspace{0.5cm}

\noindent Donc \quad \fbox{$tr(p) = r = rg(p)$}
\clearpage \subsection{Démonstration du Critère Spécial des Séries Alternées}

\color{black}
\vspace{0.3cm}

\noindent $\sum u_n$ est une série alternée.\\
$(|u_n|)_{n \in \mathbb{N}}$ $\searrow$ et converge vers 0.

\vspace{0.3cm}
\noindent Posons $\forall n \in \mathbb{N}, \quad a_n = S_{2n} = \sum_{k=0}^{2n} u_k$\\
\hspace*{3cm} $b_n = S_{2n+1} = \sum_{k=0}^{2n+1} u_k$

\vspace{0.3cm}
\noindent On suppose ici que $\forall n \in \mathbb{N}, \quad u_n (-1)^n \ge 0$.\\
Donc $|u_n| = (-1)^n u_n$.

\vspace{0.5cm}
\noindent \textbf{Montrons que $(a_n)$ et $(b_n)$ sont des suites adjacentes :}

\noindent Soit $n \in \mathbb{N}$,
\begin{itemize}
    \item $a_{n+1} - a_n = S_{2n+2} - S_{2n} = \sum_{k=2n+1}^{2n+2} u_k = u_{2n+2} + u_{2n+1}$
    \[ = |u_{2n+2}| - |u_{2n+1}| \]
    $\le 0$ \quad car $|u_n|$ est une suite décroissante.
    \noindent Donc $(a_n)$ est décroissante.

    \item $b_{n+1} - b_n = S_{2n+3} - S_{2n+1} = u_{2n+3} + u_{2n+2} \ge 0$
    \[ = -|u_{2n+3}| + |u_{2n+2}| \ge 0 \quad \text{car } |u_n| \searrow \]
    \noindent Donc $(b_n)$ est une suite croissante.

    \item $a_n - b_n = S_{2n} - S_{2n+1} = -u_{2n+1} = |u_{2n+1}| \xrightarrow{n \to +\infty} 0$
\end{itemize}

\vspace{0.3cm}
\noindent Donc $(a_n)$ et $(b_n)$ sont adjacentes.\\
Donc elles convergent vers la même limite $l$.\\
Donc $(S_n)$ converge.

\vspace{0.3cm}
\noindent \fbox{
    \begin{minipage}{\textwidth}
        \centering
        Donc \quad $\sum u_n$ converge.
    \end{minipage}
}
\clearpage \subsection*{Soit $X$ une variable aléatoire vérifiant $X(\Omega) = \mathbb{N}^*$ et d'espérance finie.}

\subsubsection{Montrer que $\frac{1}{X}$ est d'espérance finie.
}


$X(\Omega) = \mathbb{N}^*$ d'espérance finie. \\
Donc \quad $X \geqslant 1$

Donc \quad $0 \leqslant \frac{1}{X} \leqslant 1$

Donc $\frac{1}{X}$ est borné. \\
Donc $\frac{1}{X}$ est d'espérance finie.

\subsubsection{On suppose que $X$ suit une loi géométrique de paramètre $p < 1$. \\
Montrer que $\frac{1}{E(X)} \le E \left(\frac{1}{X}\right)$}



$X \hookrightarrow \mathcal{G}(p)$

D'après la formule de transfert :
\begin{align*}
    E\left(\frac{1}{X}\right) &= \sum_{n=1}^{+\infty} \frac{1}{n} P(X=n) \\
    &= \sum_{n=1}^{+\infty} \frac{1}{n} p q^{n-1} \\
    &= \frac{p}{q} \sum_{n=1}^{+\infty} \frac{q^n}{n}
\end{align*}

Or on sait que \quad $\forall x \in ]-1, 1[, \quad \sum_{n=1}^{+\infty} \frac{x^n}{n} = -\ln(1-x)$

Donc \quad $E\left(\frac{1}{X}\right) = \frac{p}{q} (-\ln(1-q)) = \frac{-p \ln(p)}{q}$

\vspace{0.5cm}

De plus $E(X) = \frac{1}{p}$

On a \quad $\frac{1}{E(X)} \leqslant E\left(\frac{1}{X}\right)$
\begin{align*}
    &\Rightarrow p \leqslant \frac{-p \ln(p)}{q} \\
    &\Leftrightarrow q \leqslant -\ln(p) \\
    &\Leftrightarrow \ln(p) \geqslant -q \\
    &\Leftrightarrow \ln(1-q) \geqslant -q
\end{align*}

\vspace{0.5cm}

On sait que \quad $\forall x > -1, \quad \ln(1+x) \leqslant x$ \\
\phantom{On sait que} \quad $\forall x > 0, \quad \ln(x) \leqslant x-1$

On a donc $\ln(1-q) \leqslant -q$

Donc \quad $\frac{1}{E(X)} \leqslant E\left(\frac{1}{X}\right)$



\subsubsection{Montrer cette inégalité dans le cas général.}



On a vu que $\frac{1}{X}$ d'espérance finie.

Posons \quad $Y = \sqrt{X}$ \quad et \quad $Z = \frac{1}{\sqrt{X}}$

\[
\begin{array}{l}
    Y \in \mathcal{M}_2(\Omega) \\
    Z \in \mathcal{M}_2(\Omega)
\end{array}
\]

D'après l'Inégalité de Cauchy Schwarz :
\[
E(YZ)^2 \leqslant E(Y^2) E(Z^2)
\]

\[
1 = E(1)^2 \leqslant E(X) E\left(\frac{1}{X}\right)
\]

\[
\frac{1}{E(X)} \leqslant E\left(\frac{1}{X}\right)
\]
\clearpage \subsection{Une urne contient des boules blanches et des boules noires. 
\\ La proportion de boules blanches est $p \in ]0, 1[$.
\\On effectue $n$ tirages avec remise d'une boule. Soit $X_n$ la var égale au nombre de boules blanches tirées.
\\Comment doit-on choisir $n$ pour affirmer avec un risque d'erreur $\le 5\%$ que $\frac{X_n}{n}$ est une valeur approchée de $p$ à $10^{-2}$ près ?}





\[
\text{On a} \quad U \left| \begin{array}{l} B \\ N \end{array} \right. \quad P(B) = p
\]

\vspace{0.5cm}

On effectue $n$ tirages avec remis. \\
$p$ est la probabilité d'avoir une boule blanche. \\
$X_n$ est le nombre de boules blanches tirées.

\[
\text{Donc} \quad X_n \hookrightarrow \mathcal{B}(n, p)
\]

\vspace{0.5cm}

On cherche $n$ tel que :
\[
P\left( \left| \frac{X_n}{n} - p \right| < 10^{-2} \right) \geqslant 0,95
\]

\begin{align*}
    \text{On pose } \quad A &= \left( \left| \frac{X_n}{n} - p \right| < 10^{-2} \right) \\
    \overline{A} &= \left( \left| \frac{X_n}{n} - p \right| \geqslant 10^{-2} \right)
\end{align*}

\vspace{0.1cm}

Posons $Z = \frac{X_n}{n}$. De plus par linéarité :
\[
\begin{cases}
    E(Z) = E(X_n) \times \frac{1}{n} \\[10pt]
    V(Z) = V(X_n) \times \frac{1}{n^2}
\end{cases}
\]

Or $X_n \hookrightarrow \mathcal{B}(n, p)$. Donc :
\[
\begin{cases}
    E(Z) = p \\[10pt]
    V(Z) = \frac{pq}{n}
\end{cases}
\]

\vspace{0.1cm}

D'après l'inégalité de Bienaymé-Tchebychev :
\[
P\left(|Z - E(Z)| \geqslant 10^{-2} \right) \leqslant \frac{V(Z)}{(10^{-2})^2} = \frac{1}{10^{-4}} \frac{pq}{n}
\]

\vspace{0.2cm}

On souhaite $P(A) \geqslant 0,95 \implies P(\overline{A}) \leqslant 0,05$.

Donc il suffit de prendre $n$ tel que :
\[
\frac{pq}{n 10^{-4}} \leqslant 0,05
\]

Donc tel que :
\[
\frac{pq 10^4}{5 \times 10^{-2}} = pq 10^6 \leqslant n
\]

\vspace{0.3cm}

\[
\text{Donc } \quad n \geqslant \lfloor pq 10^6 \rfloor + 1
\]
\clearpage \subsection{Soit $x > 0$ et soit $X$ la variable aléatoire discrète à valeurs dans $\mathbb{N}$ dont la loi de probabilité est : $\forall n \in \mathbb{N}, P(X=n) = \frac{x^{2n}}{\text{ch}(x)(2n)!}$}

\subsubsection{Vérifier que cette définition est cohérente et calculer la fonction génératrice $G_X$ (on distinguera les valeurs positives et négatives).}


\[
P(X=n) = \frac{x^{2n}}{\text{ch}(x)(2n)!}
\]

$X(\Omega) = \mathbb{N}$

On a bien $\forall n \in \mathbb{N}, \quad P(X=n) \geqslant 0$

\[
\sum_{n=0}^{+\infty} P(X=n) = \sum_{n=0}^{+\infty} \frac{x^{2n}}{\text{ch}(x)(2n)!} = \frac{1}{\text{ch}(x)} \text{ch}(x) = 1
\]

Donc c'est bien une loi de probabilité.

\underline{Si $t \geqslant 0$}

\begin{align*}
    G_X(t) &= \sum_{n=0}^{+\infty} P(X=n)t^n = \sum_{n=0}^{+\infty} \frac{x^{2n}}{\text{ch}(x)(2n)!} t^n \\
    &= \sum_{n=0}^{+\infty} \frac{(x\sqrt{t})^{2n}}{(2n)!} \frac{1}{\text{ch}(x)}
\end{align*}

\[
\boxed{G_X(t) = \frac{\text{ch}(x\sqrt{t})}{\text{ch}(x)}}
\]

\underline{Si $t \leqslant 0$}

Alors $t = -|t|$

\[
\sum_{n=0}^{+\infty} \frac{x^{2n} t^n}{(2n)!} = \sum_{n=0}^{+\infty} \frac{(-1)^n (x\sqrt{|t|})^{2n}}{(2n)!}
\]

On reconnaît \quad $\cos(u) = \sum_{n=0}^{+\infty} \frac{(-1)^n u^{2n}}{(2n)!}$

Donc \quad \boxed{G_X(t) = \frac{\cos(x\sqrt{|t|})}{\text{ch}(x)}}


\subsubsection{En déduire E(X) et V (X)}

\subsection*{b) \underline{E(X)?}}

$R = +\infty$. Donc $G_X \in \mathcal{C}^\infty(\mathbb{R})$

Donc $G_X$ est dérivable en 1.
Donc $E(X)$ existe et $E(X) = G'_X(1)$

$\forall t > 0, \quad G_X(t) = \frac{\text{ch}(x\sqrt{t})}{\text{ch}(x)}$

Donc \quad $G'_X(t) = \frac{x}{2\sqrt{t}} \frac{\text{sh}(x\sqrt{t})}{\text{ch}(x)}$

Donc \quad $E(X) = \frac{x}{2} \frac{\text{sh}(x)}{\text{ch}(x)}$

\vspace{0.5cm}

\underline{V(X)?}

$G_X$ est 2 fois dérivable en 1 et par dérivation terme à terme d'une série entière.

$\forall t > 0, \quad G''_X(t) = \sum_{n=2}^{+\infty} P(X=n) n(n-1) t^{n-2}$

En particulier en $t=1$
\[
G''_X(1) = \sum_{n=2}^{+\infty} P(X=n) n(n-1)
\]

D'après la formule de transfert on a $X(X-1)$ d'espérance finie
et $E(X^2 - X) = G''_X(1)$

$X^2 = (X^2 - X) + X$ est d'espérance finie. Donc admet une variance.

\begin{align*}
    V(X) &= E(X^2) - E(X)^2 \\
    &= E(X^2 - X) + E(X) - E(X)^2
\end{align*}

$\forall t > 0 \quad G''_X(t) = \frac{-x}{4t^{3/2}} \frac{\text{sh}(x\sqrt{t})}{\text{ch}(x)}$

\[
G''_X(1) = \frac{-x}{4} \frac{\text{sh}(x)}{\text{ch}(x)}
\]

\[
V(X) = \frac{-x}{4} \frac{\text{sh}(x)}{\text{ch}(x)} + \frac{x^2}{4} + \frac{x \text{sh}(x)}{2\text{ch}(x)} - \frac{x^2 \text{sh}^2(x)}{4\text{ch}^2(x)}
\]



\end{document}
