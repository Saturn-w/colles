\subsection{Soit $A \in S_n(\mathbb{R})$. Montrer que $A \in S_n^+(\mathbb{R})$ si et seulement si \\ il existe $B \in \mathcal{M}_n(\mathbb{R})$ telle que $A = B^T B$.}

\color{black}
\vspace{0.5cm}

\textbf{\large ($\Leftarrow$)} $\exists B \in \mathcal{M}_n(\mathbb{R})$ tq $A = B^T B$

\vspace{0.3cm}

$\bullet \ A^T = (B^T B)^T = B^T (B^T)^T = B^T B = A$ \quad Donc $A \in S_n(\mathbb{R})$

\vspace{0.3cm}

$\bullet \ $ Soit $X \in \mathbb{R}^n$
\[ X^T A X = X^T B^T B X = (BX)^T BX = \|BX\|^2 \ge 0 \]

\vspace{0.3cm}

Donc $A \in S_n^+(\mathbb{R})$

\vspace{0.8cm}

\textbf{\large ($\Rightarrow$)} $A \in S_n^+(\mathbb{R})$

\vspace{0.3cm}

$A$ est symétrique réelle donc d'après le théorème spectral : \\
$\exists P \in \mathcal{O}_n(\mathbb{R})$ tq $P^{-1} AP = D = \begin{pmatrix} \lambda_1 & & \\ & \ddots & \\ & & \lambda_n \end{pmatrix}$

\vspace{0.3cm}

Or $A \in S_n^+(\mathbb{R})$ Donc $\forall k \in \llbracket 1, n \rrbracket, \ \lambda_k \ge 0$

\vspace{0.3cm}

Posons $\Delta = \begin{pmatrix} \sqrt{\lambda_1} & & \\ & \ddots & \\ & & \sqrt{\lambda_n} \end{pmatrix} \quad$ d'où $D = \Delta^2$

\vspace{0.3cm}

$A = P \Delta^2 P^{-1} = P \Delta P^{-1} P \Delta P^{-1}$ \\
\phantom{$A$} $= P \Delta P^T P \Delta P^T \quad$ car $P \in \mathcal{O}_n(\mathbb{R})$

\vspace{0.3cm}

Si on pose $B = P \Delta P^T$ \\
Alors $B^T = B$ Donc $B$ est symétrique.

\vspace{0.5cm}

\fbox{$B^T B = A$}