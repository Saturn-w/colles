\subsection{Donner les éléments géométriques de la transformation de $\mathbb{R}^3$ dans la matrice dans la base canonique est : \\ $A = \frac{1}{3}\begin{pmatrix} 1 & 2 & -2 \\ 2 & 1 & 2 \\ -2 & 2 & 1 \end{pmatrix}$ ou $C = \begin{pmatrix} 0 & 1 & 0 \\ -1 & 0 & 0 \\ 0 & 0 & 1 \end{pmatrix}$}

\color{black}
\vspace{0.5cm}

\subsubsection{Cas de la matrice $A$}

\vspace{0.3cm}

En posant $(C_1, C_2, C_3)$ ses colonnes :

\vspace{0.3cm}

$\|C_1\|^2 = \frac{1}{3^2}(9) = 1$ \quad ; \quad $\|C_2\|^2 = \frac{1}{3^2}(4+4+1) = 1$

\vspace{0.3cm}

$(C_1 | C_2) = \frac{1}{9}(2+2-4) = 0$

\vspace{0.3cm}

$(C_1 \wedge C_2) = \frac{1}{9} \begin{pmatrix} 1 \\ 2 \\ -2 \end{pmatrix} \wedge \begin{pmatrix} 2 \\ 1 \\ 2 \end{pmatrix} = \frac{1}{9} \begin{pmatrix} 6 \\ -6 \\ -3 \end{pmatrix} = \frac{1}{3} \begin{pmatrix} 2 \\ -2 \\ -1 \end{pmatrix}$

\vspace{0.3cm}

Donc $(C_1, C_2, C_3)$ BON de $\mathbb{R}^3$. \\
Donc $f \in \mathcal{O}(E)$.

\vspace{0.3cm}

Or $A$ est symétrique. \\
Donc $f$ est une symétrie orthogonale.

\vspace{0.3cm}

$\begin{cases} A^T A = I_3 \\ A^T = A \end{cases} \implies \begin{cases} A^2 = I_3 \\ A \in \mathcal{O}(3) \end{cases} \implies \begin{cases} f \text{ sym} \\ f \text{ isom} \end{cases}$

\vspace{0.3cm}

$AX = X \iff (A-I)X = 0$ \\
$\iff \begin{pmatrix} -2 & 2 & -2 \\ 2 & -2 & 2 \\ -2 & 2 & -2 \end{pmatrix} \begin{pmatrix} x \\ y \\ z \end{pmatrix} = \begin{pmatrix} 0 \\ 0 \\ 0 \end{pmatrix}$

\vspace{0.5cm}

Donc \fbox{$f$ est la réflexion par rapport à $P = a^\perp$ avec $a = (1, -1, 1)$}

\vspace{0.8cm}

\subsubsection{Cas de la matrice $C$}

\vspace{0.3cm}

Posons $C_1, C_2, C_3$ les vecteurs colonnes de $C$. \\
$\|C_1\| = \|C_2\| = 1$ \\
$(C_1 | C_2) = 0$

\vspace{0.3cm}

$C_1 \wedge C_2 = \begin{pmatrix} 0 \\ -1 \\ 0 \end{pmatrix} \wedge \begin{pmatrix} 1 \\ 0 \\ 0 \end{pmatrix} = \begin{pmatrix} 0 \\ 0 \\ 1 \end{pmatrix} = C_3$

\vspace{0.3cm}

Donc $f \in \mathcal{SO}(\mathbb{R}^3)$. $f$ est une rotation.

\vspace{0.3cm}

\textbf{Axe :} on cherche $E_1(f)$. \\
Or $f(e_3) = e_3$ donc $e_3 \in E_1(f)$. Or $\dim(E_1(f)) = 1$. \\
Donc $E_1(f) = \mathbb{R} e_3$. On oriente l'axe par $e_3$.

\vspace{0.3cm}

\textbf{Angle :} $\text{Tr}(f) = 1 + 2 \cos \theta = 1$ \\
$\cos \theta = 0 \iff \theta \equiv \pm \pi/2 \pmod{2\pi}$

\vspace{0.3cm}

$e_1 \notin \text{Vect}(e_3)$ \\
$\text{sg} \sin \theta = \text{sg} (\text{det}(e_3, e_1, f(e_1)))$ \\
$= \text{sg} \begin{vmatrix} 0 & 1 & 0 \\ 0 & 0 & -1 \\ 1 & 0 & 0 \end{vmatrix} = -1$

\vspace{0.3cm}

Donc $\theta \equiv -\pi/2 \pmod{2\pi}$.

\vspace{0.5cm}

Donc \fbox{$f = \text{rot}_{e_3, -\pi/2}$}