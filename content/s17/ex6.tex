\subsection{Soit $f$ une isométrie d'un espace vectoriel euclidien $E$. Montrer que $f$ est \\ diagonalisable si et seulement si $f$ est une symétrie orthogonale.}

\color{black}
\vspace{0.5cm}

$f \in \mathcal{O}(E)$

\vspace{0.3cm}

$(\Leftarrow) \quad f$ est une symétrie orthogonale \\
\[ f \circ f = Id \]

\vspace{0.3cm}

Donc $X^2 - 1$ est un polynôme scindé annulateur à racines simples de $f$
\[ X^2 - 1 = (X+1)(X-1) \]
Donc $f$ est diagonalisable.

\vspace{0.8cm}

$(\Rightarrow) \quad f$ est diagonalisable \\
Donc $\exists B = (e_1, \dots, e_n)$ Base de $E$ formée de vecteurs propres de $f$
\[ mat_B(f) = \begin{pmatrix} \lambda_1 & & \\ & \ddots & \\ & & \lambda_n \end{pmatrix} \]

\vspace{0.3cm}

$\lambda_k \in Sp(f) \iff \exists x \neq 0, \enspace f(x) = \lambda x$ \\
$f$ conserve la norme donc :
\[ \|f(x)\| = \|x\| \Rightarrow |\lambda| \cdot \|x\| = \|x\| \]

\vspace{0.3cm}

Donc $|\lambda_k| = 1$. Donc $\lambda_k = \pm 1$. \\
On en déduit que $A^2 = I_n$. \\
Donc $f$ est une symétrie.

\vspace{0.3cm}

Or $f$ conserve la norme, donc :
\vspace{0.5cm}
\fbox{$f$ est une symétrie orthogonale}