\subsection{Soit $g : x \mapsto \int_0^{+\infty} \frac{Arctan(tx)}{1+t^2} dt$. }

\color{black}
\vspace{0.5cm}

\subsubsection{Montrer que $g$ est continue sur $\mathbb{R}$.}


\vspace{0.3cm}

\noindent a) $\forall x \in \mathbb{R}, \quad t \longmapsto f(x,t)$ cont sur $\mathbb{R}_+$

\vspace{0.3cm}

\noindent b) $\forall t \ge 0, \quad x \longmapsto f(x,t)$ cont sur $\mathbb{R}$ car Arctan l'est.

\vspace{0.3cm}

\noindent c) Hypo de domination : $\forall x \in \mathbb{R}, \forall t \in \mathbb{R}_+$
\[ |f(x,t)| = \left| \frac{Arctan(tx)}{1+t^2} \right| \le \frac{\pi/2}{1+t^2} = \varphi(t) \]

\vspace{0.3cm}

\noindent On a $\varphi \in L^1(\mathbb{R}_+, \mathbb{R}_+)$ \quad $\left( \varphi(t) \sim_{+\infty} \frac{\pi/2}{t^2} \right)$

\vspace{0.3cm}

\noindent Donc d'après le théo de continuité des intégrales à paramètre :

\vspace{0.3cm}

\noindent \fbox{$g$ est cont sur $\mathbb{R}$}

\subsubsection{Déterminer la limite de $g$ en $+\infty$.}


\vspace{0.3cm}

\noindent a) $\forall x \in \mathbb{R}, \quad t \longmapsto f(x,t)$ cont sur $\mathbb{R}_+$

\vspace{0.3cm}

\noindent b) $\forall t \in \mathbb{R}_+, \quad f(x,t) = \frac{Arctan(tx)}{1+t^2} \xrightarrow[x \to +\infty]{} \begin{cases} 0 & \text{si } t=0 \\ \frac{\pi/2}{1+t^2} & \text{si } t > 0 \end{cases}$

\vspace{0.3cm}

\noindent Posons $\varphi : t \longmapsto \begin{cases} \frac{\pi/2}{1+t^2} & \text{si } t > 0 \\ 0 & \text{si } t=0 \end{cases}$

\vspace{0.3cm}

\noindent $\varphi$ cpmx sur $\mathbb{R}_+$

\vspace{0.3cm}

\noindent c) HD déjà vu
\[ \forall x \in \mathbb{R}, \forall t \in \mathbb{R}_+, \quad |f(x,t)| \le \frac{\pi/2}{1+t^2} = \psi(t) \]

\vspace{0.3cm}

\noindent Donc d'après le théo de CV dominée à paramètre cont. :
\[ \lim_{x \to +\infty} g(x) = \int_0^{+\infty} \lim_{x \to +\infty} f(x,t) dt = \int_0^{+\infty} \frac{\pi/2}{1+t^2} dt \]
\[ = \frac{\pi}{2} [\text{Arctan}(t)]_0^{+\infty} \]


 \fbox{$\lim_{x \to +\infty} g(x) =  \frac{\pi^2}{4}$}