\subsection{ Soit $E$ un $\mathbb{K}$-espace vectoriel de dimension finie, $F_1, F_2, \dots, F_p$ $p$ sous-espaces de $E$. \\
Démontrer que $\dim\left(\sum_{k=1}^p F_k\right) \le \sum_{k=1}^p \dim(F_k)$ \\
et $\dim\left(\sum_{k=1}^p F_k\right) = \sum_{k=1}^p \dim(F_k)$  $\iff$ la somme est directe}

\color{black}
\vspace{0.5cm}

\noindent Posons $\psi : \begin{array}{ccl} F_1 \times \dots \times F_p & \longrightarrow & \sum_{k=1}^p F_k \\ (x_1, \dots, x_p) & \longmapsto & \sum_{k=1}^p x_k \end{array}$

\vspace{0.3cm}

\begin{itemize}
    \setlength\itemsep{0.2cm}
    \item $\psi$ est bien définie car $\forall (x_1, \dots, x_p) \in F_1 \times \dots \times F_p, \quad \sum_{k=1}^p x_k \in \sum_{k=1}^p F_k$.
    \item $\psi$ est linéaire (évident).
    \item $\text{Im } \psi = \left\{ \psi(x_1, \dots, x_p), (x_1, \dots, x_p) \in F_1 \times \dots \times F_p \right\} = \sum_{k=1}^p F_k$.
    \noindent Donc $\psi$ est surjective.
    \item $\ker \psi = \left\{ (x_1, \dots, x_p) \in (F_1 \times \dots \times F_p) \text{ tel que } \sum_{k=1}^p x_k = 0 \right\}$.
\end{itemize}

\vspace{0.5cm}

\noindent \textbf{Théorème du rang :}
\[ \dim(F_1 \times \dots \times F_p) = \dim \text{Im } \psi + \dim \ker \psi \]
\[ \sum_{k=1}^p \dim F_k = \dim \left( \sum_{k=1}^p F_k \right) \quad + \underbrace{\dim \ker \psi}_{\ge 0} \]

\vspace{0.3cm}

\noindent Donc \quad \fbox{$\dim\left(\sum_{k=1}^p F_k\right)\le \sum_{k=1}^p \dim(F_k)$ }

\vspace{0.5cm}

\noindent De plus $\sum_{k=1}^p \dim F_k = \dim \sum_{k=1}^p F_k$

\[ \iff \dim \ker \psi = 0 \]
\[ \iff \ker \psi = \{(0, \dots, 0)\} \]
\[ \iff \forall (x_1, \dots, x_p) \in F_1 \times \dots \times F_p, \quad \left( \sum_{k=1}^p x_k = 0 \implies (x_1, \dots, x_p) = (0, \dots, 0) \right) \]

\vspace{0.2cm}
\noindent $\iff$ La seule décomposition du vecteur nul est la décomposition nulle.
\noindent $\iff$ La somme est directe.