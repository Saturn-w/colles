\documentclass[a4paper,11pt]{article}
\usepackage[utf8]{inputenc}
\usepackage[T1]{fontenc}
\usepackage{amsmath, amssymb, stmaryrd}
\usepackage{xcolor}
\usepackage[margin=2.5cm]{geometry}
\usepackage{fancybox}
\usepackage{needspace}

\usepackage{tocloft}
\setlength{\cftbeforesecskip}{0.9cm}
\setlength{\cftbeforesubsecskip}{0.05cm}

\usepackage[nobottomtitles*]{titlesec}
\renewcommand{\bottomtitlespace}{2cm}

\definecolor{questionblue}{RGB}{0, 0, 205}

\pdfstringdefDisableCommands{%
  \def\int{∫}%
  \def\sum{∑}%
  \def\prod{∏}%
  \def\infty{∞}%
  \def\partial{∂}%
  \def\leq{≤}%
  \def\geq{≥}%
  \def\neq{≠}%
  \def\approx{≈}%
  \def\times{×}%
  \def\cdot{·}%
  \def\mathbb#1{#1}%
  \def\mathcal#1{#1}%
  \def\vec#1{#1}%
}

\newcommand{\sectionbreak}{\needspace{6\baselineskip}}
\newcommand{\subsectionbreak}{\needspace{10\baselineskip}}
\newcommand{\subsubsectionbreak}{\needspace{5\baselineskip}}

\titleformat{\section}
  {\normalfont\Large\bfseries\color{black}}{\thesection}{1em}{}
\titleformat{\subsection}
  {\normalfont\large\bfseries\color{questionblue}}{\thesubsection}{1em}{}
\renewcommand{\thesubsubsection}{\alph{subsubsection})}
\titleformat{\subsubsection}
  {\normalfont\large\bfseries\color{questionblue}}{\thesubsubsection}{0.5em}{}

\renewcommand{\thesubsection}{\arabic{subsection}}
\setcounter{tocdepth}{2}

\usepackage{fancyhdr}
\pagestyle{fancy}
\fancyhf{}
\fancyhead[L]{Indications - Semaine 19 - Équations différentielles}
\fancyfoot[C]{\thepage}
\renewcommand{\headrulewidth}{0pt}

\usepackage{hyperref}
\hypersetup{hidelinks}

\begin{document}

\title{Indications -- Semaine 19 -- Équations différentielles\\
\large PSI}
\date{}
\maketitle
\tableofcontents

\clearpage

% -----------------------------------------------------------------------
\subsection{$xy' - 2y = x^4$ sur $\mathbb{R}_+^*$, $\mathbb{R}_-^*$, $\mathbb{R}$}

\medskip
\textbf{Indications.}
\begin{itemize}
  \item Normaliser : $y' - \frac{2}{x} y = x^3$ pour $x \neq 0$.
  \item Solution homogène : $y_0 = \lambda x^2$ (intégrer $y'/y = 2/x$).
  \item Variation de la constante sur $\mathbb{R}_+^*$ : poser $y = \lambda(x) x^2$, obtenir $\lambda'(x) = x$, donc $\lambda(x) = x^2/2 + C$. Solution générale sur $\mathbb{R}_+^*$ : $y = Cx^2 + x^4/2$.
  \item Sur $\mathbb{R}$ : raccorder en 0. Analyser la continuité et dérivabilité de $y(x) = ax^2 + x^4/2$ (pour $x > 0$) et $y(x) = bx^2 + x^4/2$ (pour $x < 0$) en $x = 0$ : toutes les valeurs de $a, b$ sont admissibles.
\end{itemize}

\clearpage

% -----------------------------------------------------------------------
\subsection{Solutions bornées de $y' - a(t)y = 0$ quand $a \in L^1(\mathbb{R}_+)$}

\medskip
\textbf{Indications.}
\begin{itemize}
  \item Les solutions sont $y(t) = \lambda e^{A(t)}$ où $A(t) = \int_0^t a(x)\, dx$.
  \item $a$ intégrable sur $\mathbb{R}_+$ $\Rightarrow$ $A(t) \to L = \int_0^{+\infty} a(x)\, dx \in \mathbb{R}$.
  \item Donc $e^{A(t)} \to e^L$, borné. Toutes les solutions sont bornées.
  \item La réciproque est fausse en général : donner un contre-exemple si demandé.
\end{itemize}

\clearpage

% -----------------------------------------------------------------------
\subsection{$y'' + 4y = 2\sin^2(x)$}

\medskip
\textbf{Indications.}
\begin{itemize}
  \item Transformer le second membre : $2\sin^2(x) = 1 - \cos(2x)$.
  \item Équation homogène : $r^2 + 4 = 0 \Rightarrow r = \pm 2i$. Solutions homogènes : $A\cos(2x) + B\sin(2x)$.
  \item Solution particulière pour $y'' + 4y = 1$ : $y_{p1} = 1/4$.
  \item Solution particulière pour $y'' + 4y = -\cos(2x)$ : $2i$ est racine du polynôme caractéristique, donc chercher $y_{p2} = x(a\cos(2x) + b\sin(2x))$. Substituer et identifier.
  \item Solution générale = homogène + $y_{p1}$ + $y_{p2}$.
\end{itemize}

\clearpage

% -----------------------------------------------------------------------
\subsection{$y'' + xy' + y = 0$ en série entière}

\medskip
\textbf{Indications.}
\begin{itemize}
  \item Poser $y = \sum_{n \geq 0} a_n x^n$. Calculer $y'$ et $y''$, substituer dans l'équation.
  \item Identifier les coefficients : terme en $x^n$ donne $(n+2)(n+1) a_{n+2} + n a_n + a_n = 0$, soit $a_{n+2} = -\frac{a_n}{n+2}$.
  \item Deux solutions indépendantes selon la parité : parties paires (données par $a_0$, $a_1 = 0$) et impaires (données par $a_0 = 0$, $a_1$).
  \item La partie paire somme vers $e^{-x^2/2}$ (reconnaître). La partie impaire est une série entière à $R = +\infty$.
\end{itemize}

\clearpage

% -----------------------------------------------------------------------
\subsection{$x^2 y'' + y = 0$ par le changement $t = \ln x$}

\medskip
\textbf{Indications.}
\begin{itemize}
  \item Poser $z(t) = y(e^t)$. Calculer $z'(t) = e^t y'(e^t)$ et $z''(t) = e^{2t} y''(e^t) + e^t y'(e^t)$.
  \item L'équation $x^2 y'' + y = 0$ devient $z'' - z' + z = 0$ (équation à coefficients constants).
  \item Discriminant $\Delta = 1 - 4 = -3 < 0$ : $r = \frac{1 \pm i\sqrt{3}}{2}$.
  \item $z(t) = e^{t/2}(A \cos(\frac{\sqrt{3}}{2}t) + B\sin(\frac{\sqrt{3}}{2}t))$. Revenir à $y(x) = \sqrt{x}(A \cos(\frac{\sqrt{3}}{2}\ln x) + B\sin(\frac{\sqrt{3}}{2}\ln x))$.
\end{itemize}

\clearpage

% -----------------------------------------------------------------------
\subsection{Système $x' = -x + 3y + 1$, $y' = -2x + 4y$}

\medskip
\textbf{Indications.}
\begin{itemize}
  \item Mettre sous forme matricielle $X' = AX + B$ avec $A = \begin{pmatrix}-1&3\\-2&4\end{pmatrix}$, $B = \begin{pmatrix}1\\0\end{pmatrix}$.
  \item Diagonaliser $A$ : $\chi_A = \lambda^2 - 3\lambda + 2 = (\lambda-1)(\lambda-2)$, valeurs propres 1 et 2.
  \item Vecteurs propres : $\lambda_1 = 1 \Rightarrow u_1 = (3, 2)^T$, $\lambda_2 = 2 \Rightarrow u_2 = (1, 1)^T$.
  \item Poser $X = PY$ avec $P = (u_1 | u_2)$ : $Y' = DY + P^{-1}B$ donne deux EDL scalaires indépendantes.
  \item Solution particulière constante : $Y_p = -D^{-1} P^{-1} B$. Résoudre et revenir à $X$.
\end{itemize}

\end{document}
