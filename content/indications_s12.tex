\documentclass[a4paper,11pt]{article}
\usepackage[utf8]{inputenc}
\usepackage[T1]{fontenc}
\usepackage{amsmath, amssymb, stmaryrd}
\usepackage{xcolor}
\usepackage[margin=2.5cm]{geometry}
\usepackage{fancybox}
\usepackage{needspace}

\usepackage{tocloft}
\setlength{\cftbeforesecskip}{0.9cm}
\setlength{\cftbeforesubsecskip}{0.05cm}

\usepackage[nobottomtitles*]{titlesec}
\renewcommand{\bottomtitlespace}{2cm}

\definecolor{questionblue}{RGB}{0, 0, 205}

\pdfstringdefDisableCommands{%
  \def\int{∫}%
  \def\sum{∑}%
  \def\prod{∏}%
  \def\infty{∞}%
  \def\partial{∂}%
  \def\leq{≤}%
  \def\geq{≥}%
  \def\neq{≠}%
  \def\approx{≈}%
  \def\times{×}%
  \def\cdot{·}%
  \def\mathbb#1{#1}%
  \def\mathcal#1{#1}%
  \def\vec#1{#1}%
}

\newcommand{\sectionbreak}{\needspace{6\baselineskip}}
\newcommand{\subsectionbreak}{\needspace{10\baselineskip}}
\newcommand{\subsubsectionbreak}{\needspace{5\baselineskip}}

\titleformat{\section}
  {\normalfont\Large\bfseries\color{black}}{\thesection}{1em}{}
\titleformat{\subsection}
  {\normalfont\large\bfseries\color{questionblue}}{\thesubsection}{1em}{}
\renewcommand{\thesubsubsection}{\alph{subsubsection})}
\titleformat{\subsubsection}
  {\normalfont\large\bfseries\color{questionblue}}{\thesubsubsection}{0.5em}{}

\renewcommand{\thesubsection}{\arabic{subsection}}
\setcounter{tocdepth}{2}

\usepackage{fancyhdr}
\pagestyle{fancy}
\fancyhf{}
\fancyhead[L]{Indications - Semaine 12 - Séries entières}
\fancyfoot[C]{\thepage}
\renewcommand{\headrulewidth}{0pt}

\usepackage{hyperref}
\hypersetup{hidelinks}

\begin{document}

\title{Indications -- Semaine 12 -- Séries entières\\
\large PSI}
\date{}
\maketitle
\tableofcontents

\clearpage

% -----------------------------------------------------------------------
\subsection{$|a_n| \leq |b_n| \Rightarrow R_a \geq R_b$}

\medskip
\textbf{Indications.}
\begin{itemize}
  \item Soit $r < R_b$. Par définition du rayon, la suite $(b_n r^n)$ est bornée : $|b_n r^n| \leq M$.
  \item Alors $|a_n r^n| \leq |b_n r^n| \leq M$ : la suite $(a_n r^n)$ est aussi bornée.
  \item Donc $r \leq R_a$. Cela étant vrai pour tout $r < R_b$, on a $R_a \geq R_b$.
\end{itemize}

\clearpage

% -----------------------------------------------------------------------
\subsection{Développement en série entière de $e^x$}

Montrer que $e^x = \sum_{n=0}^{+\infty} \frac{x^n}{n!}$ pour tout $x \in \mathbb{R}$.

\medskip
\textbf{Indications.}
\begin{itemize}
  \item Écrire la formule de Taylor-Lagrange à l'ordre $N$ : $e^x = \sum_{n=0}^N \frac{x^n}{n!} + R_N(x)$ avec $|R_N(x)| \leq e^{|x|} \frac{|x|^{N+1}}{(N+1)!}$.
  \item La suite $\frac{|x|^{N+1}}{(N+1)!} \to 0$ (terme général d'une série convergente).
  \item Donc $R_N(x) \to 0$, ce qui donne le développement souhaité.
\end{itemize}

\clearpage

% -----------------------------------------------------------------------
\subsection{Développement en série entière de $\arctan$}

\medskip
\textbf{Indications.}
\begin{itemize}
  \item Partir de $\arctan'(x) = \frac{1}{1+x^2}$. Pour $|x| < 1$, développer en série géométrique : $\frac{1}{1+x^2} = \sum_{n=0}^{+\infty} (-1)^n x^{2n}$.
  \item Intégrer terme à terme (convergence normale sur tout $[-r, r]$ avec $r < 1$) : $\arctan(x) = \sum_{n=0}^{+\infty} \frac{(-1)^n x^{2n+1}}{2n+1}$.
  \item Le rayon de convergence est $R = 1$. La convergence en $x = \pm 1$ peut se déduire du CSSA.
\end{itemize}

\clearpage

% -----------------------------------------------------------------------
\subsection{Rayons de convergence de trois séries entières}

\medskip
\textbf{Indications.}
\begin{itemize}
  \item[(a)] $\sum \frac{\ln(n)}{n^2} x^n$ : encadrer $a_n = \ln(n)/n^2$ entre $1/n^2$ et $n/n^2 = 1/n$. Les deux séries majorante et minorante ont $R = 1$, donc $R_1 = 1$. De même pour la série associée $R_2 = 1$.
  \item[(b)] $\sum a_n z^{2n}$ avec $R_0 = 1$ : séparer parties paires et impaires. $\sum a_n z^{2n}$ converge pour $|z^2| < 1$, soit $|z| < 1$, donc $R = 1$.
  \item[(c)] $\sum \lfloor n\sqrt{5} \rfloor z^n$ : comme $\sqrt{5}$ est irrationnel, $n\sqrt{5}$ n'est jamais entier, donc $\lfloor n\sqrt{5} \rfloor \sim n\sqrt{5}$. Le terme général $a_n z^n \sim n\sqrt{5} z^n$ ne tend pas vers 0 pour $|z| = 1$, donc $R \leq 1$. D'Alembert ou comparaison donne $R = 1$.
\end{itemize}

\clearpage

% -----------------------------------------------------------------------
\subsection{Calcul de $\sum_{n=0}^{+\infty} \frac{n^2 - 1}{n!}$}

\medskip
\textbf{Indications.}
\begin{itemize}
  \item Calculer le rayon : d'Alembert, $\frac{(n+1)^2 - 1}{(n+1)!} / \frac{n^2-1}{n!} = \frac{n^2+2n}{(n+1)(n^2-1)} \to 0$, donc $R = +\infty$.
  \item Décomposer : $n^2 - 1 = n(n-1) + n - 1$.
  \item $\sum \frac{n(n-1)}{n!} = \sum \frac{1}{(n-2)!} = e$, $\sum \frac{n}{n!} = e$, $\sum \frac{1}{n!} = e$.
  \item Calculer la somme : $\sum \frac{n^2-1}{n!} = e + e - e = e$.
\end{itemize}

\clearpage

% -----------------------------------------------------------------------
\subsection{Rayon de $\sum (a_n)^2 z^n$}

Si $\sum a_n z^n$ a un rayon $R$, quel est le rayon $R'$ de $\sum (a_n)^2 z^n$ ?

\medskip
\textbf{Indications.}
\begin{itemize}
  \item \textbf{Minoration $R' \geq R^2$ :} si $|z| < R^2$, choisir $r$ tel que $|z| < r^2 < R^2$, i.e. $r < R$. Alors $(a_n r^n)$ est bornée par $M$, donc $|a_n^2 z^n| = (a_n r^n)^2 (z/r^2)^n \leq M^2 (|z|/r^2)^n \to 0$.
  \item \textbf{Majoration $R' \leq R^2$ :} si $|z| > R^2$, montrer que $(a_n^2 z^n)$ n'est pas bornée en trouvant des $n$ pour lesquels $|a_n|$ est grand, via la définition du rayon $R$.
  \item Conclusion : $R' = R^2$.
\end{itemize}

\clearpage

% -----------------------------------------------------------------------
\subsection{DSE de $\mathrm{ch}(x)\cos(x)$}

\medskip
\textbf{Indications.}
\begin{itemize}
  \item Écrire $\mathrm{ch}(x) \cos(x) = \frac{e^x + e^{-x}}{2} \cos(x) = \mathrm{Re}\!\left(\frac{e^{x(1+i)} + e^{x(-1+i)}}{2}\right)$.
  \item $e^{x(1+i)} = \sum \frac{(1+i)^n x^n}{n!}$. Or $(1+i) = \sqrt{2} e^{i\pi/4}$, donc $(1+i)^n = 2^{n/2} e^{in\pi/4}$.
  \item $\mathrm{Re}(e^{x(1+i)}) + \mathrm{Re}(e^{x(-1+i)})$ : les termes impairs disparaissent, et on ne garde que les puissances $x^{4k}$ (les termes $x^{4k+2}$ s'annulent aussi).
  \item Résultat : $\mathrm{ch}(x)\cos(x) = \sum_{k=0}^{+\infty} \frac{(-1)^k 2^{2k}}{(4k)!} x^{4k}$.
\end{itemize}

\end{document}
