\subsection{E un $\mathbb{K}$-EV, $h \in \mathcal{L}(E)$ tel que $rg(h)=1$ \\
\null\hfill Montrer que $h^2 = tr(h)h$ \hfill\null}

\color{black}
\vspace{0.5cm}

\noindent D'après le théorème du rang
\[ \dim E = \dim \ker h + rg(h) \]
\noindent Donc \quad $\dim \ker h = n-1$

\vspace{0.5cm}

\noindent Soit $B_K = (e_1, \dots, e_{n-1})$ une base de $\ker h$
\noindent D'après le théorème de la base incomplète
\noindent Cette famille se complète en une base de $E$
\[ B = (e_1, \dots, e_{n-1}, e_n) \]

\vspace{0.5cm}

\noindent tel que \quad $mat_B(h) = \left( \begin{array}{c|c} \huge{0} & \begin{matrix} x_1 \\ \vdots \\ x_n \end{matrix} \end{array} \right) = H$

\vspace{0.5cm}

\noindent On a $tr(h) = x_n$

\vspace{0.2cm}

\noindent De plus \quad $H^2 = \left( \begin{array}{c|c} \huge{0} & \begin{matrix} x_1 x_n \\ \vdots \\ x_n x_n \end{matrix} \end{array} \right)$

\vspace{0.3cm}

\[ H^2 = x_n H \]

\vspace{0.5cm}

\noindent Donc \quad \fbox{$h^2 = tr(h)h$}