\subsection{Inégalité de Cauchy Schwarz. Cas d'égalité.}
\color{black}
\vspace{0.5cm}

$\forall x,y \in E \ , \ \lambda \in \mathbb{R} \ ,$

\vspace{0.3cm}

\[
\begin{aligned}
\|x + \lambda y\|^2 &= (x + \lambda y | x + \lambda y) \\
&= (x|x) + \lambda^2 (y|y) + 2\lambda (x|y)
\end{aligned}
\]

\vspace{0.3cm}

On reconnaît un Polynôme du $2^{\text{nd}}$ degré en $\lambda$

\vspace{0.3cm}

De plus $\|x + \lambda y\| \geqslant 0$

\vspace{0.3cm}

Donc $\Delta \leqslant 0$

\vspace{0.3cm}

Donc $4 (x|y)^2 \leqslant 4 \|x\|^2 \|y\|^2$

\vspace{0.3cm}

Donc \fbox{$|(x|y)| \leqslant \|x\| \|y\|$}

\vspace{0.3cm}

cas d'égalité :

\vspace{0.3cm}

si $|(x|y)| = \|x\| \|y\|$

\vspace{0.3cm}

Alors $\Delta = 0$

\vspace{0.3cm}

Donc $\exists \lambda_0 \in \mathbb{R}$ tel que $P(\lambda_0)=0$

\vspace{0.3cm}

\[ \| x + \lambda_0 y\|^2 = 0 \]

\vspace{0.3cm}

Par séparation

\vspace{0.3cm}

$x + \lambda_0 y = 0$

\vspace{0.3cm}

$x = -\lambda_0 y$

\vspace{0.3cm}

\fbox{$x$ et $y$ sont colinéaires.}