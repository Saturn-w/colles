\subsection{Soit $p$ projecteur. \\  $p$ projecteur orthogonal $\iff \forall x \in E, \|p(x)\| \leq \|x\|$}
\color{black}
\vspace{0.5cm}

 $(\Leftarrow)$
\color{black}
\vspace{0.5cm}

Montrons que $\text{Im}(p)$ et $\ker(p)$ orthogonaux.

\vspace{0.3cm}
Soit $x \in \ker(p)$, $y \in \text{Im}(p)$
\vspace{0.3cm}
$\forall \lambda \in \mathbb{R}, \|p(x+\lambda y)\|^2 \leq \|x+\lambda y\|^2$

\vspace{0.3cm}
Donc $\|\lambda p(y)\|^2 \leq \|x + \lambda y\|^2$ \\
(avec $p(y)=y$)

\vspace{0.3cm}
Donc $\lambda^2 \|y\|^2 \leq \|x\|^2 + \lambda^2 \|y\|^2 + 2\lambda (x|y)$

\vspace{0.3cm}
Donc $\forall \lambda \in \mathbb{R}, \quad 0 \leq \|x\|^2 + 2\lambda (x|y)$

\vspace{0.3cm}
Si $(x|y)$ non nul, alors $\|x\|^2 + 2\lambda (x|y)$ change de signe \\
(Polynôme de degré 1) \\
Impossible

\vspace{0.3cm}
Donc $(x|y) = 0$

\vspace{0.3cm}
\fbox{Donc $\ker(p)$ et $\text{Im } p$ sont orthogonaux}

$(\Rightarrow)$
\color{black}
\vspace{0.5cm}

$p$ est un projecteur orthogonal \\
$p$ projection sur $\text{Im}(p)$ parallèlement à $\ker(p)=(\text{Im } p)^\perp$ \\
Donc $\ker(p)$ est orthogonal à $\text{Im } p$

\vspace{0.3cm}
Soit $x \in E$, $\exists ! (x_I, x_{I^\perp}) \in \text{Im}(p) \times \text{Im}(p)^\perp$ tel que $x = x_I + x_{I^\perp}$

\vspace{0.3cm}
$p(x) = x_I \quad \text{Donc } \|p(x)\|^2 = \|x_I\|^2$

\vspace{0.3cm}
D'après le théorème de pythagore \\
$\|x\|^2 = \|x_I\|^2 + \|x_{I^\perp}\|^2$

\vspace{0.3cm}
Donc $\|p(x)\|^2 \leq \|x\|^2$

\vspace{0.3cm}
\fbox{Donc $\|p(x)\| \leq \|x\|$}