\documentclass[a4paper,11pt]{article}
\usepackage[utf8]{inputenc}
\usepackage[T1]{fontenc}
\usepackage{amsmath, amssymb, stmaryrd}
\usepackage{xcolor}
\usepackage[margin=2.5cm]{geometry}
\usepackage{fancybox}
\usepackage{needspace}

\usepackage{tocloft}
\setlength{\cftbeforesecskip}{0.9cm}
\setlength{\cftbeforesubsecskip}{0.05cm}

\usepackage[nobottomtitles*]{titlesec}
\renewcommand{\bottomtitlespace}{2cm}

\definecolor{questionblue}{RGB}{0, 0, 205}

\pdfstringdefDisableCommands{%
  \def\int{∫}%
  \def\sum{∑}%
  \def\prod{∏}%
  \def\infty{∞}%
  \def\partial{∂}%
  \def\leq{≤}%
  \def\geq{≥}%
  \def\neq{≠}%
  \def\approx{≈}%
  \def\times{×}%
  \def\cdot{·}%
  \def\mathbb#1{#1}%
  \def\mathcal#1{#1}%
  \def\vec#1{#1}%
}

\newcommand{\sectionbreak}{\needspace{6\baselineskip}}
\newcommand{\subsectionbreak}{\needspace{10\baselineskip}}
\newcommand{\subsubsectionbreak}{\needspace{5\baselineskip}}

\titleformat{\section}
  {\normalfont\Large\bfseries\color{black}}{\thesection}{1em}{}
\titleformat{\subsection}
  {\normalfont\large\bfseries\color{questionblue}}{\thesubsection}{1em}{}
\renewcommand{\thesubsubsection}{\alph{subsubsection})}
\titleformat{\subsubsection}
  {\normalfont\large\bfseries\color{questionblue}}{\thesubsubsection}{0.5em}{}

\renewcommand{\thesubsection}{\arabic{subsection}}
\setcounter{tocdepth}{2}

\usepackage{fancyhdr}
\pagestyle{fancy}
\fancyhf{}
\fancyhead[L]{Indications - Semaine 4 - Algèbre linéaire}
\fancyfoot[C]{\thepage}
\renewcommand{\headrulewidth}{0pt}

\usepackage{hyperref}
\hypersetup{hidelinks}

\begin{document}

\title{Indications -- Semaine 4 -- Algèbre linéaire\\
\large PSI}
\date{}
\maketitle
\tableofcontents

\clearpage

% -----------------------------------------------------------------------
\subsection{$\mathrm{tr}(AB) = \mathrm{tr}(BA)$}

\medskip
\textbf{Indications.}
\begin{itemize}
  \item Développer les coefficients diagonaux : $[AB]_{ii} = \sum_k a_{ik} b_{ki}$.
  \item Donc $\mathrm{tr}(AB) = \sum_i \sum_k a_{ik} b_{ki}$.
  \item Permuter les indices $i$ et $k$ pour reconnaître $\mathrm{tr}(BA) = \sum_k \sum_i b_{ki} a_{ik}$.
\end{itemize}

\clearpage

% -----------------------------------------------------------------------
\subsection{$\mathrm{tr}(p) = \mathrm{rg}(p)$ pour un projecteur}

\medskip
\textbf{Indications.}
\begin{itemize}
  \item Un projecteur vérifie $p^2 = p$, donc $E = \mathrm{Im}\, p \oplus \ker p$.
  \item Choisir une base adaptée à cette décomposition : dans cette base, la matrice de $p$ est $\begin{pmatrix} I_r & 0 \\ 0 & 0 \end{pmatrix}$ où $r = \mathrm{rg}(p)$.
  \item La trace vaut $r$ dans cette base, et la trace est invariante par changement de base.
\end{itemize}

\clearpage

% -----------------------------------------------------------------------
\subsection{Déterminant de Vandermonde}

Calculer $V_n = \det\begin{pmatrix} 1 & a_1 & \cdots & a_1^{n-1} \\ \vdots & & & \vdots \\ 1 & a_n & \cdots & a_n^{n-1} \end{pmatrix} = \prod_{1 \leq i < j \leq n} (a_j - a_i)$.

\medskip
\textbf{Indications.}
\begin{itemize}
  \item Raisonnement par récurrence sur $n$.
  \item Fixer $a_1, \dots, a_{n-1}$ et considérer $P(x) = V(a_1, \dots, a_{n-1}, x)$ comme un polynôme de degré $n-1$ en $x$ (développer selon la dernière ligne).
  \item $P(a_i) = 0$ pour $i = 1, \dots, n-1$ (deux lignes égales), donc $P(x) = C \prod_{i=1}^{n-1}(x - a_i)$.
  \item Identifier le coefficient dominant $C = V_{n-1}$, puis conclure par récurrence.
\end{itemize}

\clearpage

% -----------------------------------------------------------------------
\subsection{$\mathrm{rg}(h) = 1 \Rightarrow h^2 = \mathrm{tr}(h) \cdot h$}

\medskip
\textbf{Indications.}
\begin{itemize}
  \item $\mathrm{rg}(h) = 1$ : $\ker h$ est de dimension $n-1$. Choisir une base $(e_1, \dots, e_{n-1}, e_n)$ où $(e_1, \dots, e_{n-1})$ est une base de $\ker h$.
  \item Dans cette base, la matrice $H$ a toutes ses colonnes nulles sauf la dernière : $H = (0 | \cdots | 0 | v)$ où $v = h(e_n)$.
  \item Calculer $H^2$ : $H^2 e_i = 0$ pour $i < n$, et $H^2 e_n = H(v) = v_n \cdot v$ où $v_n$ est la $n$-ième coordonnée de $v$.
  \item Identifier $v_n = \mathrm{tr}(H) = \mathrm{tr}(h)$.
\end{itemize}

\clearpage

% -----------------------------------------------------------------------
\subsection{La famille $((k^n)_{n \geq 0})_{k \geq 0}$ est libre dans $\mathbb{R}^\mathbb{N}$}

Montrer que toute sous-famille finie est libre.

\medskip
\textbf{Indications.}
\begin{itemize}
  \item Considérer une relation $\sum_{k=0}^{N} \lambda_k (k^n)_{n \geq 0} = 0$, i.e. $\sum_{k=0}^N \lambda_k k^n = 0$ pour tout $n \geq 0$.
  \item Évaluer en $n = 0, 1, \dots, N$ : on obtient le système $V \cdot \lambda = 0$ où $V$ est la matrice de Vandermonde associée aux points $0, 1, \dots, N$.
  \item Le déterminant de Vandermonde est $\prod_{0 \leq i < j \leq N}(j - i) \neq 0$, donc $\lambda = 0$.
\end{itemize}

\end{document}
