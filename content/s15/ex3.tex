\subsection{Démonstration de l'Inégalité de Markov }

\color{black}
\vspace{0.5cm}


- $X$ v.a.d positive, d'espérance finie \\ 
- Soit $a > 0$ \\ 
Montrer que $P(X \ge a) \le \frac{E(X)}{a}$
\vspace{0.3cm}

\noindent $E(X) = \sum_{x \in X(\Omega)} x P(X=x) \quad (x P(X=x))_{x \in X(\Omega)} \text{ famille sommable}$

\vspace{0.3cm}

\noindent On peut sommer par paquets :
\[ E(X) = \sum_{\substack{x \in X(\Omega) \\ x \ge a}} x P(X=x) + \sum_{\substack{x \in X(\Omega) \\ x < a}} x P(X=x) \]

\vspace{0.3cm}

\noindent $\bullet$ Or $X \ge 0$ donc $x \ge 0$, d'où $\sum_{\substack{x \in X(\Omega) \\ x < a}} x P(X=x) \ge 0$

\vspace{0.3cm}

\noindent $\bullet$ On a donc :
\[ E(X) = \sum_{\substack{x \in X(\Omega) \\ x \ge a}} x P(X=x) + \sum_{\substack{x \in X(\Omega) \\ x < a}} x P(X=x)  \ge a \sum_{\substack{x \in X(\Omega) \\ x \ge a}} P(X=x) + 0 \]

\vspace{0.3cm}

\noindent Or $\sum_{\substack{x \in X(\Omega) \\ x \ge a}} P(X=x) = P(X \ge a)$

\vspace{0.3cm}

\noindent Donc $E(X) \ge a P(X \ge a) \quad$ car $a > 0$

\vspace{0.5cm}

\noindent Donc \quad \fbox{$P(X \ge a) \le \frac{E(X)}{a}$}