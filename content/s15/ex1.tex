\subsection{Démonstration de la Variance d'une loi de Poisson $X \sim \mathcal{P}(\lambda)$ .}

\color{black}
\vspace{0.5cm}
Soit $X \sim \mathcal{P}(\lambda)$. On sait que $E(X)=\lambda$. \\ 
Démontrer que $X^2$ est d'espérance finie et calculer $V(X)$
\vspace{0.4cm}

\noindent $X^2$ est d'espérance finie $\iff \sum k^2 P(X=k)$ ACV (CV car termes positifs) \\
\noindent \phantom{$X^2$ est d'espérance finie} $\iff \sum k^2 \frac{\lambda^k}{k!} e^{-\lambda}$

\vspace{0.3cm}

\noindent On sait que $\forall x \in \mathbb{R}, \quad e^x = \sum_{k=0}^{+\infty} \frac{x^k}{k!}$

\vspace{0.3cm}

\noindent Par dérivation t à t d'une SE \quad $e^x = \sum_{k=1}^{+\infty} \frac{k x^{k-1}}{k!}$

\vspace{0.3cm}

\noindent On multiplie par $x$ et on redérive \quad $x e^x = \sum_{k=0}^{+\infty} \frac{k x^k}{k!}$ \\
\noindent \phantom{On multiplie par $x$ et on redérive} $e^x(1+x) = \sum_{k=1}^{+\infty} \frac{k^2 x^{k-1}}{k!}$

\vspace{0.3cm}

\noindent Donc $\sum \frac{k^2}{k!} \lambda^k$ CV on en déduit que $X^2$ est d'espérance finie

\vspace{0.3cm}

\noindent et $E(X^2) = \sum_{k=0}^{+\infty} k^2 P(X=k) = \sum_{k=1}^{+\infty} \frac{k^2 \lambda^{k-1}}{k!} \times \lambda e^{-\lambda}$ \\
\noindent \phantom{et $E(X^2)$} $= (e^\lambda(1+\lambda)) \lambda e^{-\lambda}$ \\
\noindent \phantom{et $E(X^2)$} $= \lambda + \lambda^2$

\vspace{0.8cm}

\noindent D'après la formule de Koenig Huygens : \\
\noindent $V(X) = E(X^2) - E(X)^2 = \lambda + \lambda^2 - \lambda^2 = \lambda$

\vspace{0.5cm}

\noindent Donc \fbox{$V(X) = \lambda$}