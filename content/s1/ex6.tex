
% Partie Titre / Question en bleu
\noindent \color{questionblue}
\subsection{ Soit $n \in \mathbb{N}^*$. Montrer que \\
$D_n(\mathbb{R}) = \text{Vect}(D^k, k \in \llbracket 0, n-1 \rrbracket)$ \\
où $D = \begin{pmatrix} 1 & \dots & 0 \\ \vdots & \ddots & \vdots \\ 0 & \dots & n \end{pmatrix}$.}

% Partie Réponse en noir
\color{black}
\vspace{0.4cm}

\noindent \textbf{(M1)} \textbf{Montrons que $\{ D^k, k \in \llbracket 0, n-1 \rrbracket \}$ est une famille libre.}
\\\noindent Soit $(\lambda_0, \dots, \lambda_{n-1}) \in \mathbb{R}^n$ tel que $\sum_{k=0}^{n-1} \lambda_k D^k = 0$.

\noindent Posons $P = \sum_{k=0}^{n-1} \lambda_k X^k$.
\[ P(D) = 0 \implies \begin{pmatrix} P(1) & & (0) \\ & \ddots & \\ (0) & & P(n) \end{pmatrix} = \begin{pmatrix} 0 & & \\ & \ddots & \\ & & 0 \end{pmatrix} \]

\noindent \\Donc le polynôme a $n$ racines différentes $1, \dots, n$ et est de degré $n-1$.
\noindent Donc c'est le polynôme nul. Et $\forall k \in \llbracket 0, n-1 \rrbracket, \lambda_k = 0$.
\noindent Donc la famille est libre.

\noindent $\rightarrow$ On a alors une famille libre de $n$ éléments de $D_n(\mathbb{R})$.
Or $\dim(D_n(\mathbb{R})) = n$.
\noindent Donc c'est une base.
\[ \text{Donc } \quad \text{Vect}(D^k, k \in \llbracket 0, n-1 \rrbracket) = D_n(\mathbb{R}) \]

\vspace{0.8cm}

\noindent \textbf{(M2)} \textbf{On pose $F = \text{Vect}\{ D^k, k \in \llbracket 0, n-1 \rrbracket \}$.}
\begin{itemize}
    \item $\forall k \in \llbracket 0, n-1 \rrbracket, D^k \in D_n(\mathbb{R})$. Donc $F \subset D_n(\mathbb{R})$.
    \item \textbf{Montrons que $\forall M \in D_n(\mathbb{R}), M \in F$.}
\end{itemize}
Soit $M = \begin{pmatrix} \alpha_1 & & 0 \\ & \ddots & \\ 0 & & \alpha_n \end{pmatrix} \in D_n(\mathbb{R})$.

\noindent Montrons qu'il existe $(\lambda_0, \dots, \lambda_{n-1})$ tel que $M = \sum_{k=0}^{n-1} \lambda_k D^k$.
\[ \text{càd } \begin{pmatrix} \alpha_1 & & \\ & \ddots & \\ & & \alpha_n \end{pmatrix} = \begin{pmatrix} \sum \lambda_k 1^k & & \\ & \ddots & \\ & & \sum \lambda_k n^k \end{pmatrix} \]

\noindent Posons $\psi : \mathbb{R}_{n-1}[X] \longrightarrow \mathbb{R}^n, \quad P \longmapsto (P(1), \dots, P(n))$.
\noindent C'est un isomorphisme (cours).

\noindent Donc $\exists! P \in \mathbb{R}_{n-1}[X]$ tel que $(\alpha_1, \dots, \alpha_n) = (P(1), \dots, P(n)) = \psi(P)$.
\\ \noindent Donc $M = \begin{pmatrix} P(1) & & \\ & \ddots & \\ & & P(n) \end{pmatrix} = P(D)$.

\noindent Si on note $P = \sum_{k=0}^{n-1} \lambda_k X^k$, alors $M \in F$.
\noindent Par double inclusion \fbox{$D_n(\mathbb{R}) = F$}.

\vspace{0.8cm}

\noindent \textbf{(M3)} \textbf{Soit $(L_1, \dots, L_n)$ la famille des polynômes d'interpolation de Lagrange associée à $(1, \dots, n)$.}

\noindent Soit $k \in \llbracket 1, n \rrbracket$.
\[ L_k(D) = \begin{pmatrix} L_k(1) & & (0) \\ & \ddots & \\ (0) & & L_k(n) \end{pmatrix} = \begin{pmatrix} 0 & & & \\ & 1 & & \\ & & \ddots & \\ & & & 0 \end{pmatrix} = E_{kk} \]

\noindent $L_k(D)$ est un polynôme sur $D$, donc combinaison linéaire de $(D^0, \dots, D^{n-1})$.
 \\ \noindent Donc $\forall k, E_{kk} \in F$.

  \noindent Or $(E_{kk})_{k \in \llbracket 1, n \rrbracket}$ est une base de $D_n(\mathbb{R})$.
 \\ \noindent Donc $D_n(\mathbb{R}) \subset F$.
\\ \noindent Autre inclusion évidente.
\\ \noindent Donc par double inclusion \fbox{$D_n(\mathbb{R}) = F$}.

