\subsection{Démonstration de la règle de d'Alembert ($l < 1$)}

\color{black}
\vspace{0.3cm}

\noindent Soit $(u_n)$ une suite strictement positive.
\[ \lim_{n \to +\infty} \frac{u_{n+1}}{u_n} = l < 1 \]

\noindent \textbf{Montrons que $\sum u_n$ converge.}

\vspace{0.3cm}
\noindent Soit $\alpha \in ]l, 1[$. Posons $\varepsilon = \alpha - l > 0$.
\[ \exists n_0 \in \mathbb{N}, \forall n \ge n_0, \quad l - \varepsilon \le \frac{u_{n+1}}{u_n} \le l + \varepsilon = \alpha \]

\noindent Or $u_n > 0$. Donc $\forall n \ge n_0, \quad u_{n+1} \le \alpha u_n$.

\vspace{0.3cm}
\noindent On cherche à montrer par récurrence simple que $\mathcal{P}(n) : "u_n \le \alpha^{n-n_0} u_{n_0}"$.

\begin{itemize}
    \item \textbf{Initialisation à $n_0$ :}
    \[ u_{n_0} \le \alpha^{n_0 - n_0} u_{n_0} = 1 \times u_{n_0} \quad \text{(Vrai)} \]

    \item \textbf{Hérédité :}
    Soit $n \ge n_0$. Supposons $\mathcal{P}(n)$.
    \noindent On sait que $u_{n+1} \le \alpha u_n$.
    \noindent D'après l'hypothèse de récurrence :
    \[ u_{n+1} \le \alpha u_n \le \alpha (\alpha^{n-n_0} u_{n_0}) \]
    \[ u_{n+1} \le \alpha^{n+1-n_0} u_{n_0} \]
    \noindent Donc $\mathcal{P}(n+1)$ est vérifiée.
\end{itemize}

\noindent Donc $\forall n \ge n_0, \quad u_n \le \alpha^{n-n_0} u_{n_0}$.

\vspace{0.3cm}

\noindent On a donc $\forall n \ge n_0, \quad 0 \le u_n \le \alpha^n (\alpha^{-n_0} u_{n_0})$.
\[ \text{Donc } \quad u_n = O(\alpha^n) \]

\noindent Or $\sum \alpha^n$ converge (série géométrique car $|\alpha| < 1$, ici $0 < \alpha < 1$).

\vspace{0.2cm}
\noindent D'après le théorème de comparaison des séries à termes positifs :


\noindent \fbox{
    \begin{minipage}{\textwidth}
        \centering
        $\sum u_n$ converge.
    \end{minipage}
}