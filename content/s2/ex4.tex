\subsection{Convergence d'une suite d'entiers ($(u_n) \in \mathbb{Z}^{\mathbb{N}}$)}

\color{black}
\vspace{0.3cm}

\noindent \textbf{Question :} Soit $(u_n)_{n \in \mathbb{N}} \in \mathbb{Z}^{\mathbb{N}}$.\\
Montrer que $(u_n)$ converge $\iff (u_n)$ est stationnaire.

\vspace{0.5cm}

\noindent \textbf{($\Leftarrow$)} Si $(u_n)$ est stationnaire.
\[ \exists n_0 \in \mathbb{N}, \forall n \ge n_0, \quad u_n = u_{n_0} \]
Donc $(u_n)$ converge vers $u_{n_0}$.

\vspace{0.5cm}

\noindent \textbf{($\Rightarrow$)} Si $(u_n)_{n \in \mathbb{N}}$ converge vers $l$ sa limite.
\noindent Alors pour $\varepsilon = \frac{1}{4}$.
\[ \exists n_0 \in \mathbb{N}, \forall n \ge n_0, \quad |u_n - l| \le \frac{1}{4} \]

\noindent On regarde la distance entre deux termes au-delà de $n_0$ :
\[ |u_n - u_{n_0}| \le |u_n - l| + |l - u_{n_0}| \le \frac{1}{4} + \frac{1}{4} = \frac{1}{2} \]

\noindent Or, $(u_n) \in \mathbb{Z}^{\mathbb{N}}$, donc la différence $u_n - u_{n_0}$ est un entier relatif.
\noindent Un entier dont la valeur absolue est inférieure ou égale à $1/2$ est nécessairement nul.
\[ |u_n - u_{n_0}| \le \frac{1}{2} \implies |u_n - u_{n_0}| = 0 \]

\noindent Donc $\forall n \ge n_0, \quad u_n = u_{n_0}$.

\vspace{0.3cm}
\noindent \fbox{
    \begin{minipage}{\textwidth}
        \centering
        Donc la suite est stationnaire.
    \end{minipage}
}