\subsection{Étude de la suite $u_{n+1} = \frac{1}{2+u_n}$}

\color{black}
\vspace{0.3cm}

\noindent \textbf{Énoncé :} Étudier la suite $\begin{cases} u_0 \in [0, 1] \\ \forall n \in \mathbb{N}, u_{n+1} = \frac{1}{2+u_n} \end{cases}$

\vspace{0.3cm}

\noindent On pose $f : x \longmapsto \frac{1}{2+x}$ et $I = [0, 1]$.

\noindent Sur $I$, $f$ est décroissante :
\[
\begin{array}{c|ccc}
 x & 0 & & 1 \\
 \hline
 f & 1/2 & \searrow & 1/3
\end{array}
\]

\noindent $f$ stabilise $I$. Donc $(u_n)$ est bien définie et bornée car $\forall n, u_n \in [0, 1]$.

\vspace{0.5cm}

\noindent $f$ continue sur $I$. \textbf{Point fixe :}
\[ f(x) = x \iff \frac{1}{2+x} = x \iff 1 = 2x + x^2 \]
\[ \Delta = 4+4=8 \quad x = \frac{-2 \pm \sqrt{8}}{2} = -1 \pm \sqrt{2} \]
\noindent Dans $I$, il n'y a que le point fixe $\alpha = \sqrt{2} - 1$.

\vspace{0.5cm}

\noindent \textbf{Étude de $(u_{2n})$ :}
Posons $\forall n \in \mathbb{N}, w_n = u_{2n}$. Alors $w_{n+1} = f \circ f(w_n)$.
\noindent Or $f \circ f$ est croissante ($f$ décroissante). Donc $(w_n)$ est monotone.
\noindent Or $(w_n)$ est bornée (sur $[0, 1]$).

\noindent Donc $(w_n)$ converge, on pose $\beta = \lim_{n \to +\infty} w_n$.
\noindent $\beta$ est un point fixe de $f \circ f$.

\[ \forall x \in [0, 1], \quad f \circ f(x) = \frac{1}{2+f(x)} = \frac{1}{2+\frac{1}{2+x}} = \frac{2+x}{5+2x} \]

\[ \text{Donc } f \circ f(x) = x \iff 2+x = 5x + 2x^2 \iff 0 = 2x^2 + 4x - 2 \iff x^2 + 2x - 1 = 0 \]
\[ \iff x = \sqrt{2} - 1 = \alpha \]

\noindent Donc $(u_{2n})$ converge vers $\alpha$.

\vspace{0.3cm}
\noindent \fbox{
    \begin{minipage}{\textwidth}
        \centering
        On en déduit que $(u_n)$ converge vers $\alpha$.
    \end{minipage}
}