\documentclass[a4paper,11pt]{article}
\usepackage[utf8]{inputenc}
\usepackage[T1]{fontenc}
\usepackage{amsmath, amssymb, stmaryrd}
\usepackage{xcolor}
\usepackage[margin=2.5cm]{geometry}
\usepackage{fancybox}
\usepackage{needspace}

\usepackage{tocloft}
\setlength{\cftbeforesecskip}{0.9cm} % Ajuste la valeur (ex: 1em, 15pt, etc.)

\setlength{\cftbeforesubsecskip}{0.05cm}

\usepackage[nobottomtitles*]{titlesec} % On ajoute l'option à l'import
\renewcommand{\bottomtitlespace}{2cm}

% Définition des couleurs
\definecolor{questionblue}{RGB}{0, 0, 205}

%cache les rectangles rouge sde sliens

% Sécurité : Définit l'espace nécessaire AVANT chaque titre
\newcommand{\sectionbreak}{\needspace{6\baselineskip}}
\newcommand{\subsectionbreak}{\needspace{10\baselineskip}}
\newcommand{\subsubsectionbreak}{\needspace{5\baselineskip}}


% Configuration des titres pour correspondre à ton style (1.1 en Bleu)
% Sections (Semaines)
\titleformat{\section}
  {\normalfont\Large\bfseries\color{black}}{\thesection}{1em}{}
  % Subsections (Exercices) 
\titleformat{\subsection}
  {\normalfont\large\bfseries\color{questionblue}}{\thesubsection}{1em}{}
% Subsubsections (Questions) 
\renewcommand{\thesubsubsection}{\alph{subsubsection})}
\titleformat{\subsubsection}
  {\normalfont\large\bfseries\color{questionblue}}{\thesubsubsection}{0.5em}{}


%\pretocmd{\section}{\addtocontents{toc}{\protect\filbreak}}{}{}



% Redéfinition de la numérotation des sous-sections
\renewcommand{\thesubsection}{\arabic{subsection}}
\setcounter{tocdepth}{2}

% Hyperref chargé en dernier
\usepackage{hyperref}
\hypersetup{hidelinks}

% Définir les symboles Unicode pour les bookmarks PDF
\pdfstringdefDisableCommands{%
  \def\int{∫}%
  \def\sum{∑}%
  \def\prod{∏}%
  \def\infty{∞}%
  \def\partial{∂}%
  \def\leq{≤}%
  \def\geq{≥}%
  \def\neq{≠}%
  \def\approx{≈}%
  \def\times{×}%
  \def\cdot{·}%
  \def\mathbb#1{#1}%
  \def\mathcal#1{#1}%
  \def\vec#1{#1}%
}

% Configuration de l'en-tête avec fancyhdr
\usepackage{fancyhdr}
\pagestyle{fancy}
\fancyhf{} % Efface tous les en-têtes et pieds de page
\fancyhead[L]{Semaine 2 - Suites et séries numériques} % En-tête à gauche
\fancyfoot[C]{\thepage} % Numéro de page au centre du pied de page
\renewcommand{\headrulewidth}{0pt} % Supprime la ligne de séparation

\begin{document}

\title{Semaine 2 - Suites et séries numériques\\
\large PSI}
\date{}
\maketitle
\tableofcontents

\clearpage \subsection{ Démonstration du théorème des Accroissements finis. \\
$f$ continue sur  $[a, b]$,  dérivable sur $]a, b[$ \\
Montrons que  $$\exists c \in ]a, b[, f(b) - f(a) = f'(c)(b-a)$$}

\color{black}
\vspace{0.5cm}

\noindent Posons $\varphi : x \longmapsto f(x) - f(a) - k(x-a)$

\vspace{0.2cm}
\noindent On choisit $k$ tel que $\varphi(b) = 0$ :
\noindent $\varphi(b) = 0 \implies f(b) - f(a) = k(b-a)$
\[ \implies \quad k = \frac{f(b)-f(a)}{b-a} \]

\vspace{0.5cm}

\noindent Par hypothèse sur $f$, $\varphi$ est continue sur $[a, b]$, dérivable sur $]a, b[$

\vspace{0.3cm}

\noindent $\varphi(b) = 0$ et $\varphi(a) = 0$
\noindent $\varphi(a) = \varphi(b)$ \quad D'après le théorème de Rolle
\noindent $\exists c \in ]a, b[, \quad \varphi'(c) = 0$

\vspace{0.5cm}

\noindent Or $\forall x \in ]a, b[, \quad \varphi'(x) = f'(x) - k$
\noindent Donc \quad $k = f'(c)$

\vspace{0.5cm}

\noindent Donc \quad \fbox{$\exists c \in ]a, b[, \quad f(b) - f(a) = f'(c)(b-a)$}
\clearpage \subsection{ Montrer que si $p$ projecteur alors $tr(p) = rg(p)$}

\color{black}
\vspace{0.5cm}

\noindent $p$ projecteur Donc $Im p \oplus \ker(p) = E$

\vspace{0.5cm}

\noindent Posons $r = rg(p) = \dim(Im(p))$
\noindent Soit $\mathcal{B} = (e_1, \dots, e_r, e_{r+1}, \dots, e_n)$ une base adaptée à cette décomposition.

\vspace{0.5cm}

\noindent Alors $mat_{\mathcal{B}}(p) = \left( \begin{array}{ccc|c} 1 & & 0 & \\ & \ddots & & 0 \\ 0 & & 1 & \\ \hline & 0 & & 0 \end{array} \right)$

\vspace{0.3cm}

\noindent $\forall j \in \llbracket 1, r \rrbracket, \quad e_j \in Im(p) \implies p(e_j) = e_j$
\noindent $\forall j \in \llbracket r+1, n \rrbracket, \quad e_j \in \ker(p) \implies p(e_j) = 0$

\vspace{0.5cm}

\noindent $mat_{\mathcal{B}}(p) = \left( \begin{array}{c|c} I_r & 0 \\ \hline 0 & 0 \end{array} \right)$

\vspace{0.5cm}

\noindent Donc \quad \fbox{$tr(p) = r = rg(p)$}
\clearpage \subsection{Démonstration du Critère Spécial des Séries Alternées}

\color{black}
\vspace{0.3cm}

\noindent $\sum u_n$ est une série alternée.\\
$(|u_n|)_{n \in \mathbb{N}}$ $\searrow$ et converge vers 0.

\vspace{0.3cm}
\noindent Posons $\forall n \in \mathbb{N}, \quad a_n = S_{2n} = \sum_{k=0}^{2n} u_k$\\
\hspace*{3cm} $b_n = S_{2n+1} = \sum_{k=0}^{2n+1} u_k$

\vspace{0.3cm}
\noindent On suppose ici que $\forall n \in \mathbb{N}, \quad u_n (-1)^n \ge 0$.\\
Donc $|u_n| = (-1)^n u_n$.

\vspace{0.5cm}
\noindent \textbf{Montrons que $(a_n)$ et $(b_n)$ sont des suites adjacentes :}

\noindent Soit $n \in \mathbb{N}$,
\begin{itemize}
    \item $a_{n+1} - a_n = S_{2n+2} - S_{2n} = \sum_{k=2n+1}^{2n+2} u_k = u_{2n+2} + u_{2n+1}$
    \[ = |u_{2n+2}| - |u_{2n+1}| \]
    $\le 0$ \quad car $|u_n|$ est une suite décroissante.
    \noindent Donc $(a_n)$ est décroissante.

    \item $b_{n+1} - b_n = S_{2n+3} - S_{2n+1} = u_{2n+3} + u_{2n+2} \ge 0$
    \[ = -|u_{2n+3}| + |u_{2n+2}| \ge 0 \quad \text{car } |u_n| \searrow \]
    \noindent Donc $(b_n)$ est une suite croissante.

    \item $a_n - b_n = S_{2n} - S_{2n+1} = -u_{2n+1} = |u_{2n+1}| \xrightarrow{n \to +\infty} 0$
\end{itemize}

\vspace{0.3cm}
\noindent Donc $(a_n)$ et $(b_n)$ sont adjacentes.\\
Donc elles convergent vers la même limite $l$.\\
Donc $(S_n)$ converge.

\vspace{0.3cm}
\noindent \fbox{
    \begin{minipage}{\textwidth}
        \centering
        Donc \quad $\sum u_n$ converge.
    \end{minipage}
}
\clearpage \subsection*{Soit $X$ une variable aléatoire vérifiant $X(\Omega) = \mathbb{N}^*$ et d'espérance finie.}

\subsubsection{Montrer que $\frac{1}{X}$ est d'espérance finie.
}


$X(\Omega) = \mathbb{N}^*$ d'espérance finie. \\
Donc \quad $X \geqslant 1$

Donc \quad $0 \leqslant \frac{1}{X} \leqslant 1$

Donc $\frac{1}{X}$ est borné. \\
Donc $\frac{1}{X}$ est d'espérance finie.

\subsubsection{On suppose que $X$ suit une loi géométrique de paramètre $p < 1$. \\
Montrer que $\frac{1}{E(X)} \le E \left(\frac{1}{X}\right)$}



$X \hookrightarrow \mathcal{G}(p)$

D'après la formule de transfert :
\begin{align*}
    E\left(\frac{1}{X}\right) &= \sum_{n=1}^{+\infty} \frac{1}{n} P(X=n) \\
    &= \sum_{n=1}^{+\infty} \frac{1}{n} p q^{n-1} \\
    &= \frac{p}{q} \sum_{n=1}^{+\infty} \frac{q^n}{n}
\end{align*}

Or on sait que \quad $\forall x \in ]-1, 1[, \quad \sum_{n=1}^{+\infty} \frac{x^n}{n} = -\ln(1-x)$

Donc \quad $E\left(\frac{1}{X}\right) = \frac{p}{q} (-\ln(1-q)) = \frac{-p \ln(p)}{q}$

\vspace{0.5cm}

De plus $E(X) = \frac{1}{p}$

On a \quad $\frac{1}{E(X)} \leqslant E\left(\frac{1}{X}\right)$
\begin{align*}
    &\Rightarrow p \leqslant \frac{-p \ln(p)}{q} \\
    &\Leftrightarrow q \leqslant -\ln(p) \\
    &\Leftrightarrow \ln(p) \geqslant -q \\
    &\Leftrightarrow \ln(1-q) \geqslant -q
\end{align*}

\vspace{0.5cm}

On sait que \quad $\forall x > -1, \quad \ln(1+x) \leqslant x$ \\
\phantom{On sait que} \quad $\forall x > 0, \quad \ln(x) \leqslant x-1$

On a donc $\ln(1-q) \leqslant -q$

Donc \quad $\frac{1}{E(X)} \leqslant E\left(\frac{1}{X}\right)$



\subsubsection{Montrer cette inégalité dans le cas général.}



On a vu que $\frac{1}{X}$ d'espérance finie.

Posons \quad $Y = \sqrt{X}$ \quad et \quad $Z = \frac{1}{\sqrt{X}}$

\[
\begin{array}{l}
    Y \in \mathcal{M}_2(\Omega) \\
    Z \in \mathcal{M}_2(\Omega)
\end{array}
\]

D'après l'Inégalité de Cauchy Schwarz :
\[
E(YZ)^2 \leqslant E(Y^2) E(Z^2)
\]

\[
1 = E(1)^2 \leqslant E(X) E\left(\frac{1}{X}\right)
\]

\[
\frac{1}{E(X)} \leqslant E\left(\frac{1}{X}\right)
\]
\clearpage \subsection{ Montrer que la famille des $((k^n)_{n \in \mathbb{N}})_{k \in \mathbb{N}}$ est une famille libre de $\mathbb{R}^{\mathbb{N}}$ }

\color{black}
\vspace{0.5cm}

\noindent Posons $\forall k \in \mathbb{N}, \quad u_k = (k^n)_{n \in \mathbb{N}} \in \mathbb{R}^{\mathbb{N}}$

\vspace{0.2cm}
\noindent Soit $(\lambda_0, \dots, \lambda_N)$ tel que \quad $\sum_{k=0}^N \lambda_k u_k = 0$

\vspace{0.5cm}
\noindent \null\hfill égalité de suite valable $\forall n \in \mathbb{N}$ \hfill\null

\vspace{0.2cm}

\noindent $\forall n \in \mathbb{N}, \quad \sum_{k=0}^N \lambda_k u_k(n) = 0 \quad = \sum_{k=0}^N \lambda_k k^n = 0$

\vspace{0.8cm}

\noindent $\rightarrow n=0  , \quad \lambda_0 + \lambda_1 + \dots + \lambda_N = 0$

\noindent $\rightarrow   n=1 \quad , \quad \lambda_0 \times 0 + \lambda_1 + 2\lambda_2 + \dots + N\lambda_N = 0$

 $\dots \dots$ 

\noindent $\rightarrow n=N , \quad \lambda_0 \times 0^N + \lambda_1 \times 1^N + \dots + N^N \lambda_N = 0$

\vspace{0.5cm}

\noindent On obtient le système
\[ \underbrace{\begin{pmatrix} 1 & \dots & 1 \\ 0 & 1 \dots & N \\ \vdots & & \vdots \\ 0^N & 2^N \dots & N^N \end{pmatrix}}_{A} \underbrace{\begin{pmatrix} \lambda_0 \\ \vdots \\ \lambda_N \end{pmatrix}}_{B} = \begin{pmatrix} 0 \\ \vdots \\ 0 \end{pmatrix} \]

\vspace{0.5cm}

\noindent $A$ est une matrice de Vandermonde

\noindent Donc $\det A = V(0, \dots, N)$ 

\noindent Donc $N+1$ valeurs $2$ à $2 \neq$

\vspace{0.2cm}
\noindent $\det A \neq 0 \quad$ Alors $A$ est inversible

\vspace{0.5cm}

\noindent Donc $\begin{pmatrix} \lambda_0 \\ \vdots \\ \lambda_N \end{pmatrix} = \begin{pmatrix} 0 \\ \vdots \\ 0 \end{pmatrix}$

\vspace{0.5cm}

\noindent Donc $(u_k)_{0 \le k \le N}$ est une famille libre
\noindent Donc \quad \fbox{$(u_k)_{k \in \mathbb{N}}$ est libre car toutes ses familles finies le sont}
\clearpage \subsection{Résoudre le système : $\begin{cases} x' = -x + 3y + 1 \\ y' = -2x + 4y \end{cases}$}

\color{black}
\vspace{0.5cm}

\noindent Posons $X(t) = \begin{pmatrix} x(t) \\ y(t) \end{pmatrix} \quad A = \begin{pmatrix} -1 & 3 \\ -2 & 4 \end{pmatrix} \quad B = \begin{pmatrix} 1 \\ 0 \end{pmatrix}$

\vspace{0.3cm}

\noindent $X$ sol de $(S) \iff \forall t \in \mathbb{R}, \enspace X'(t) = AX(t) + B$

\vspace{0.3cm}

\noindent $\bullet \quad \chi_A(X) = X^2 - Tr(A)X + \det(A)$ \\
\noindent \phantom{$\bullet \quad \chi_A(X)$} $= X^2 - 3X + 2$

\vspace{0.3cm}

\noindent Donc $Sp(A) = \{1, 2\}$

\vspace{0.3cm}

\noindent $A$ est une matrice de taille 2, elle a 2 vap différentes. \\
\noindent Donc $A$ est diagonalisable.

\vspace{0.3cm}

\noindent Cherchons une base de $\mathbb{R}^2$ formée de vecteurs propres.

\vspace{0.3cm}

\noindent $E_1(A) = \text{Ker}(A - I) = \text{Ker}\begin{pmatrix} -2 & 3 \\ -2 & 3 \end{pmatrix}$. On a $u_1 = \begin{pmatrix} 3 \\ 2 \end{pmatrix} \in E_1(A)$

\vspace{0.3cm}

\noindent $E_2(A) = \text{Ker}(A - 2I) = \text{Ker}\begin{pmatrix} -3 & 3 \\ -2 & 2 \end{pmatrix}$. On a $u_2 = \begin{pmatrix} 1 \\ 1 \end{pmatrix} \in E_2(A)$

\vspace{0.3cm}

\noindent $\mathcal{B} = (u_1, u_2)$ Base de $\mathbb{R}^2$ formée de vep. \\
\noindent $P = Pass_{Bc \to B} = \begin{pmatrix} 3 & 1 \\ 2 & 1 \end{pmatrix}$

\vspace{0.3cm}

\noindent On a $P^{-1}AP = D = \begin{pmatrix} 1 & 0 \\ 0 & 2 \end{pmatrix}$

\vspace{0.3cm}

\noindent $X$ sol de $(S) \iff X = PY \iff Y' = DY + P^{-1}B$

\vspace{0.3cm}

\noindent Or $P^{-1} = \frac{1}{\det P} \begin{pmatrix} 1 & -1 \\ -2 & 3 \end{pmatrix} = \begin{pmatrix} 1 & -1 \\ -2 & 3 \end{pmatrix}$

\vspace{0.3cm}

\noindent $P^{-1}B = \begin{pmatrix} 1 & -1 \\ -2 & 3 \end{pmatrix} \begin{pmatrix} 1 \\ 0 \end{pmatrix} = \begin{pmatrix} 1 \\ -2 \end{pmatrix}$

\vspace{0.3cm}

\noindent Sol de $(S) \iff \begin{cases} y'_1 = y_1 + 1 \\ y'_2 = 2y_2 - 2 \end{cases}$

\vspace{0.3cm}

\noindent $\iff \begin{cases} y_1(t) = C_1 e^t - 1 \\ y_2(t) = C_2 e^{2t} + 1 \end{cases} \quad C_1, C_2 \in \mathbb{R}$

\vspace{0.3cm}

\noindent $\iff X(t) = \begin{pmatrix} 3 & 1 \\ 2 & 1 \end{pmatrix} \begin{pmatrix} C_1 e^t - 1 \\ C_2 e^{2t} + 1 \end{pmatrix}$

\vspace{0.3cm}

\noindent $\iff \begin{pmatrix} x(t) \\ y(t) \end{pmatrix} = \begin{pmatrix} 3(C_1 e^t - 1) + (C_2 e^{2t} + 1) \\ 2(C_1 e^t - 1) + (C_2 e^{2t} + 1) \end{pmatrix}$

\vspace{0.3cm}

\noindent $\iff \begin{pmatrix} x(t) \\ y(t) \end{pmatrix} = \begin{pmatrix} 3C_1 e^t + C_2 e^{2t} - 3 + 1 \\ 2C_1 e^t + C_2 e^{2t} - 2 + 1 \end{pmatrix}$

\vspace{0.5cm}

\noindent Donc \quad \fbox{$\begin{cases} x(t) = 3C_1 e^t + C_2 e^{2t} - 2 \\ y(t) = 2C_1 e^t + C_2 e^{2t} - 1 \end{cases} \quad C_1, C_2 \in \mathbb{R}$}
\clearpage \subsection{Déterminer le DSE au voisinage de 0 de \\ $f(x) = \text{ch}(x) \cos(x)$.}

\color{black}
\vspace{0.5cm}

\noindent \textbf{Méthode Somme de DSE}

\vspace{0.3cm}

\noindent $\forall x \in \mathbb{R}, \quad \text{ch}(x) = \frac{1}{2}(e^x + e^{-x})$

\vspace{0.3cm}

\noindent $f(x) = \frac{1}{2} ( e^x \cos(x) + e^{-x} \cos(x) )$ \\
\noindent $f(x) = \frac{1}{2} ( h(x) + h(-x) )$

\vspace{0.3cm}

\noindent $\forall x \in \mathbb{R}, \quad h(x) = \text{Re}( e^x e^{ix} ) = \text{Re}( e^{x(1+i)} )$ \\
\noindent \phantom{$\forall x \in \mathbb{R}, \quad h(x)$} $= \text{Re}\left( \sum_{n=0}^{+\infty} \frac{x^n (1+i)^n}{n!} \right)$

\vspace{0.3cm}

\noindent $(1+i) = \sqrt{2} \left( \frac{1}{\sqrt{2}} + i \frac{1}{\sqrt{2}} \right) = \sqrt{2} e^{i \pi / 4}$

\vspace{0.3cm}

\noindent Donc $h(x) = \sum_{n=0}^{+\infty} \text{Re}\left( \frac{(\sqrt{2})^n e^{i (\frac{n \pi}{4})} x^n}{n!} \right)$ \\
\noindent \phantom{Donc $h(x)$} $= \sum_{n=0}^{+\infty} \frac{(\sqrt{2})^n \cos\left(\frac{n \pi}{4}\right) x^n}{n!}$

\vspace{0.3cm}

\noindent $f(x) = \frac{1}{2} \sum_{n=0}^{+\infty} \frac{(\sqrt{2})^n \cos(n \pi / 4)}{n!} (x^n + (-x)^n)$

\vspace{0.3cm}

\noindent Or \quad $x^n + (-x)^n = \begin{cases} 0 & \text{si } n \text{ impair} \\ 2x^n & \text{sinon} \end{cases}$

\vspace{0.3cm}

\noindent $f(x) = \sum_{\substack{p=0 \\ n=2p}}^{+\infty} \frac{2^p \cos\left( \frac{p \pi}{2} \right)}{(2p)!} x^{2p}$

\vspace{0.3cm}

\noindent Or \quad $\cos\left( \frac{p \pi}{2} \right) = \begin{cases} 0 & \text{si } p \text{ impair} \\ (-1)^{n'} & \text{sinon} \end{cases}$

\vspace{0.3cm}

\noindent $f(x) = \sum_{\substack{n'=0 \\ p=2n'}}^{+\infty} \frac{2^{2n'} (-1)^{n'} x^{4n'}}{(4n')!}$

\vspace{0.5cm}

\noindent Donc $\forall x \in \mathbb{R}$, \quad \fbox{$f(x) = \sum_{n=0}^{+\infty} \frac{4^n (-1)^n}{(4n)!} x^{4n}$}


\addtocontents{toc}{\protect\newpage}


\end{document}
