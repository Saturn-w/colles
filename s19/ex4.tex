\subsection{Résoudre $y'' + xy' + y = 0$ en cherchant des solutions développables en série entière.}

\color{black}
\vspace{0.5cm}

\noindent $(E) \quad y'' + xy' + y = 0$

\vspace{0.3cm}

\noindent On pose $y(x) = \sum a_n x^n$ avec $R > 0$.

\vspace{0.3cm}

\noindent $y$ solution de $(E)$ sur $]-R, R[$ \\
\noindent $\iff \forall x \in ]-R, R[, \quad \sum_{n=2}^{+\infty} a_n n(n-1) x^{n-2} + x \sum_{n=1}^{+\infty} a_n n x^{n-1} + \sum_{n=0}^{+\infty} a_n x^n = 0$ \\
\noindent $\iff \forall x \in ]-R, R[, \quad \sum_{n=0}^{+\infty} a_{n+2} (n+2)(n+1) x^n + \sum_{n=0}^{+\infty} n a_n x^n + \sum_{n=0}^{+\infty} a_n x^n = 0$

\vspace{0.3cm}

\noindent $\iff \forall n \in \mathbb{N}$, par unicité des coefficients d'une série entière :
\[ a_{n+2}(n+2)(n+1) + n a_n + a_n = 0 \]

\vspace{0.3cm}

\noindent $\iff \forall n \in \mathbb{N}, \quad a_{n+2} = -\frac{(n+1)}{(n+2)(n+1)} a_n$ \\
\noindent $\iff \forall n \in \mathbb{N}, \quad a_{n+2} = -\frac{a_n}{n+2} \quad (*)$

\vspace{0.8cm}

\noindent \textbf{Analyse : Si $y$ sol de $(E)$}

\vspace{0.3cm}

\noindent Si $n = 2p$,
\[ a_{2p} = -\frac{a_{2p-2}}{2p} = \frac{-1}{2p} \times \frac{-1}{2(p-1)} a_{2(p-1)} \]
\[ = \frac{-1}{2p} \times \frac{-1}{2(p-1)} \times \frac{-1}{2(p-2)} \times \dots \times \frac{-1}{2 \times 1} a_0 \]
\[ \fbox{$a_{2p} = \frac{(-1)^p}{2^p p!} a_0$} \]

\vspace{0.3cm}

\noindent Si $n = 2p+1$,
\[ a_{2p+1} = \frac{-a_{2p-1}}{2p+1} = \frac{-1}{2p+1} \times \frac{-1}{2p-1} \times \dots \times \frac{-1}{3} a_1 \]
\[ a_{2p+1} = \frac{(-1)^p}{(2p+1)(2p-1)\dots 1} a_1 \]
\[ \fbox{$a_{2p+1} = \frac{(-1)^p 2^p p!}{(2p+1)!} a_1$} \]

\vspace{0.5cm}

\noindent Donc \quad $y(x) = a_0 \left( \sum_{p=0}^{+\infty} \frac{(-1)^p}{2^p p!} x^{2p} \right) + a_1 \left( \sum_{p=0}^{+\infty} \frac{(-1)^p 2^p p!}{(2p+1)!} x^{2p+1} \right)$

\vspace{0.3cm}

\noindent $y$ solution de $(E)$ sur $]-R, R[$ avec les coefficients vérifiant $(*)$.

\vspace{0.8cm}

\noindent \textbf{Rayon de convergence ?}

\vspace{0.3cm}

\noindent $\bullet \quad \sum_{p=0}^{+\infty} \frac{(-1)^p}{2^p p!} x^{2p} = \sum_{p=0}^{+\infty} \frac{1}{p!} \left( \frac{-x^2}{2} \right)^p = e^{-x^2/2}$ \\
\noindent Série exp, CV $\forall x \in \mathbb{R}$. \\
\noindent $R_1 = +\infty$.

\vspace{0.3cm}

\noindent $\bullet \quad \sum_{p=0}^{+\infty} \frac{(-1)^p 2^p p!}{(2p+1)!} x^{2p+1}$ \\
\noindent Posons $u_p = \left| \frac{(-1)^p 2^p p!}{(2p+1)!} x^{2p+1} \right| = \frac{2^p p! |x|^{2p+1}}{(2p+1)!} > 0 \quad \text{si } x \neq 0$ \\
\noindent $\frac{u_{p+1}}{u_p} = \frac{2 (p+1) |x|^2}{(2p+3)(2p+2)} \xrightarrow[p \to +\infty]{} 0$

\vspace{0.3cm}

\noindent D'après la règle de d'Alembert, la série converge $\forall x \in \mathbb{R}$. \\
\noindent Donc $R_2 = +\infty$.

\vspace{0.3cm}

\noindent Donc il y a un Rayon de convergence $R = +\infty$.

\vspace{0.3cm}

\noindent Posons $\varphi_1 : x \longmapsto e^{-x^2/2}$ \\
\noindent $\varphi_2 : x \longmapsto \sum_{p=0}^{+\infty} \frac{(-1)^p 2^p p!}{(2p+1)!} x^{2p+1}$

\vspace{0.8cm}

\noindent $\bullet \quad \varphi_1, \varphi_2$ sont sol de $E$ sur $\mathbb{R}$.

\vspace{0.3cm}

\noindent $\bullet \quad (\varphi_1, \varphi_2)$ famille libre car $\varphi_1$ paire et $\varphi_2$ impaire.

\vspace{0.3cm}

\noindent $\bullet \quad (E)$ est une EDL du $2^{nd}$ ordre homogène, normalisée à coefficients continus sur $\mathbb{R}$. \\
\noindent Donc $sol_{\mathbb{R}}(E)$ est un $\mathbb{R}$-ev de dim 2.

\vspace{0.5cm}

\noindent Donc \quad \fbox{$sol_{\mathbb{R}}(E) = \text{Vect}(\varphi_1, \varphi_2)$}